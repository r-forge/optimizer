\documentclass[11pt]{article}\usepackage[]{graphicx}\usepackage[]{color}
%% maxwidth is the original width if it is less than linewidth
%% otherwise use linewidth (to make sure the graphics do not exceed the margin)
\makeatletter
\def\maxwidth{ %
  \ifdim\Gin@nat@width>\linewidth
    \linewidth
  \else
    \Gin@nat@width
  \fi
}
\makeatother

\definecolor{fgcolor}{rgb}{0.345, 0.345, 0.345}
\newcommand{\hlnum}[1]{\textcolor[rgb]{0.686,0.059,0.569}{#1}}%
\newcommand{\hlstr}[1]{\textcolor[rgb]{0.192,0.494,0.8}{#1}}%
\newcommand{\hlcom}[1]{\textcolor[rgb]{0.678,0.584,0.686}{\textit{#1}}}%
\newcommand{\hlopt}[1]{\textcolor[rgb]{0,0,0}{#1}}%
\newcommand{\hlstd}[1]{\textcolor[rgb]{0.345,0.345,0.345}{#1}}%
\newcommand{\hlkwa}[1]{\textcolor[rgb]{0.161,0.373,0.58}{\textbf{#1}}}%
\newcommand{\hlkwb}[1]{\textcolor[rgb]{0.69,0.353,0.396}{#1}}%
\newcommand{\hlkwc}[1]{\textcolor[rgb]{0.333,0.667,0.333}{#1}}%
\newcommand{\hlkwd}[1]{\textcolor[rgb]{0.737,0.353,0.396}{\textbf{#1}}}%
\let\hlipl\hlkwb

\usepackage{framed}
\makeatletter
\newenvironment{kframe}{%
 \def\at@end@of@kframe{}%
 \ifinner\ifhmode%
  \def\at@end@of@kframe{\end{minipage}}%
  \begin{minipage}{\columnwidth}%
 \fi\fi%
 \def\FrameCommand##1{\hskip\@totalleftmargin \hskip-\fboxsep
 \colorbox{shadecolor}{##1}\hskip-\fboxsep
     % There is no \\@totalrightmargin, so:
     \hskip-\linewidth \hskip-\@totalleftmargin \hskip\columnwidth}%
 \MakeFramed {\advance\hsize-\width
   \@totalleftmargin\z@ \linewidth\hsize
   \@setminipage}}%
 {\par\unskip\endMakeFramed%
 \at@end@of@kframe}
\makeatother

\definecolor{shadecolor}{rgb}{.97, .97, .97}
\definecolor{messagecolor}{rgb}{0, 0, 0}
\definecolor{warningcolor}{rgb}{1, 0, 1}
\definecolor{errorcolor}{rgb}{1, 0, 0}
\newenvironment{knitrout}{}{} % an empty environment to be redefined in TeX

\usepackage{alltt}
\usepackage{graphicx} %If you want to include postscript graphics
\usepackage{wrapfig} 
\usepackage[round]{natbib}
\usepackage{url}

\raggedbottom
\newcommand{\B}[1]{{\bf #1 \rm}}
\newcommand{\Splus}{{\sf S-PLUS}}
\newcommand{\Slang}{{\sf S}}
\newcommand{\R}{{\sf R}} 
\newcommand{\I}[1]{{\it#1\rm}}
\newcommand{\code}[1]{{\tt#1}}
\newcommand{\pkg}[1]{\bf{\tt#1}\rm }
\IfFileExists{upquote.sty}{\usepackage{upquote}}{}
\begin{document}
\SweaveOpts{concordance=TRUE}




?? find new reference for RQ times
?? put in eigen
?? clean up warnings


\begin{knitrout}
\definecolor{shadecolor}{rgb}{0.969, 0.969, 0.969}\color{fgcolor}\begin{kframe}


{\ttfamily\noindent\color{warningcolor}{\#\# Warning in dep\_auto(): corrupt dependency files? \\\#\# try remove cache/\_\_objects; cache/\_\_globals}}\end{kframe}
\end{knitrout}


\title{Optimization problems constrained by parameter sums}
\author{Gabor Grothendieck, GKX Group,\\
John C. Nash, Telfer School of Management, University of Ottawa, and\\
Ravi Varadhan, Johns Hopkins University Medical School}
\date{November 2016}
\maketitle

\section*{Abstract}

This article presents a discussion of optimization problems where the 
objective function $f(\B{x})$ has parameters that are constrained by some
scaling, so that $q(\B{x}) = constant$, where this function $q()$ involves
a sum of the parameters, their squares, or similar simple function.

\section{Background}

We consider problems where we want to minimize or maximize a function subject to a constraint
that the sum of some function of the parameters, e.g., their sum of squares, must 
equal some constant.
We refer to these problems as \B{sumscale} optimization problems. We have observed questions 
about problems like this on the R-help mailing list: 

\begin{verbatim}
Jul 19, 2012 at 10:24 AM, Linh Tran <Tranlm@berkeley.edu> wrote:
> Hi fellow R users,
>
> I am desperately hoping there is an easy way to do this in R.
>
> Say I have three functions:
>
> f(x) = x^2
> f(y) = 2y^2
> f(z) = 3z^2
>
> constrained such that x+y+z=c (let c=1 for simplicity).
>
> I want to find the values of x,y,z that will minimize 
f(x) + f(y) + f(z).
\end{verbatim}

If the parameters $x$, $y$ and $z$ are non-negative, this problem can actually 
be solved as a Quadratic Program. We revisit this problem at the end of this
article.

Other examples of this type of objective function are:

\begin{itemize}
\item{The maximum volume  of a regular polyhedron where the sum of the lengths
of the sides is fixed.}
\item{The minimum negative log likelihood for a multinomial model.}
%% ?? May want to expand
%% this -- Gabor's example is not data dependent, but it would be nice to have one of these.
\item{The Rayleigh Quotient for the maximal or minimal eigensolutions of a matrix, where
the eigenvectors should be normalized so the square norm of the vector is 1.}
\end{itemize}

For the moment, let us consider a basic example, which is 
\vspace*{10pt}

\B{Problem A}: Minimize  $( - \prod{\B{x}})$ subject to $\sum{\B{x}}=1$
\vspace*{10pt}

This is a very simplified version of the multinomial maximum likelihood problem. 

Because these problems all have an objective that is dependent on a scaled set of parameters 
where the scale is defined by a sum, sum of squares, or similar sum of the paramters, we will
refer to them as \B{sumscale} optimization problems.

\section{Difficulties using general optimization with sumscale problems}

Let us use the basic example above to consider how we might formulate Problem A for a
computational solution in \R. 

One possibility is to select one of the parameters and solve for it in 
terms of the others. Let this
be the last parameter $x_n$, so that the set of parameters to be 
optimized is $ \B{y} = (x_1, x_1, ..., x_{n-1})$ where 
$n$ is the original size of our problem. We now have the unconstrained problem
\vspace*{10pt}

$ minimize ( - (\prod{\B{y}}) * (1 - \sum{y} ) ) $

\vspace*{10pt}
This is easily coded and tried. We will use a very simple start, namely, the sequence $1,2, ...,
(n-1)$ scaled by $1/n^2$. We will also specify that the gradient is to be computed by a 
central approximation \citep{optextras}.

\begin{knitrout}\scriptsize
\definecolor{shadecolor}{rgb}{0.969, 0.969, 0.969}\color{fgcolor}\begin{kframe}
\begin{alltt}
\hlkwd{cat}\hlstd{(}\hlstr{"try loading optimrx\textbackslash{}n"}\hlstd{)}
\end{alltt}
\begin{verbatim}
## try loading optimrx
\end{verbatim}
\begin{alltt}
\hlkwd{require}\hlstd{(optimx,} \hlkwc{quietly}\hlstd{=}\hlnum{TRUE}\hlstd{)}
\hlstd{pr} \hlkwb{<-} \hlkwa{function}\hlstd{(}\hlkwc{y}\hlstd{) \{}
\hlopt{-} \hlkwd{prod}\hlstd{(y)}\hlopt{*}\hlstd{(}\hlnum{1}\hlopt{-}\hlkwd{sum}\hlstd{(y))}
\hlstd{\}}
\hlkwd{cat}\hlstd{(}\hlstr{"test the simple product for n=5\textbackslash{}n"}\hlstd{)}
\end{alltt}
\begin{verbatim}
## test the simple product for n=5
\end{verbatim}
\begin{alltt}
\hlstd{meth} \hlkwb{<-} \hlkwd{c}\hlstd{(}\hlstr{"Nelder-Mead"}\hlstd{,} \hlstr{"BFGS"}\hlstd{)}
\hlstd{n}\hlkwb{<-}\hlnum{5}
  \hlstd{st}\hlkwb{<-}\hlnum{1}\hlopt{:}\hlstd{(n}\hlopt{-}\hlnum{1}\hlstd{)}\hlopt{/}\hlstd{(n}\hlopt{*}\hlstd{n)}
   \hlstd{ans}\hlkwb{<-}\hlkwd{opm}\hlstd{(st, pr,} \hlkwc{gr}\hlstd{=}\hlstr{"grcentral"}\hlstd{,} \hlkwc{control}\hlstd{=}\hlkwd{list}\hlstd{(}\hlkwc{trace}\hlstd{=}\hlnum{0}\hlstd{))}
   \hlstd{ao}\hlkwb{<-}\hlkwd{summary}\hlstd{(ans,}\hlkwc{order}\hlstd{=value)}
\hlkwd{print}\hlstd{(ao)}
\end{alltt}
\begin{verbatim}
##                    p1        p2        p3        p4    value fevals gevals
## BFGS        0.2000000 0.1999986 0.2000037 0.1999986 -0.00032    102     96
## Nelder-Mead 0.2000034 0.1999983 0.2000017 0.2000021 -0.00032    331     NA
##             convergence kkt1 kkt2 xtime
## BFGS                  0 TRUE TRUE 0.006
## Nelder-Mead           0 TRUE TRUE 0.002
\end{verbatim}
\end{kframe}
\end{knitrout}


While these codes work fine for small $n$, it is fairly easy to see that there are 
computational problems as the size of the problem increases. Since the sum of the 
parameters is constrained to be equal to 1, the parameters are of the order of $1/n$,
and the function therefore of the order of $1/(n^n)$, which underflows around $n=144$ in 
\R. 


\section{Other formulations}

Traditionally, statisticians solve maximum likelihood problems by \B{minimizing} the 
negative log-likelihood. That is, the objective function is formed as (-1) times the
logarithm of the likelihood. This converts our product to a sum. Choosing the first
parameter to be the one determined by the summation constraint, we can write the 
function and gradient quite easily. As programs that try to find the minimum may change
the parameters so that logarithms of non-positive numbers are attempted, we have put
some safeguards in the function \code{nll}. At this point we have assumed the gradient
calculation is only attempted if the function can be computed satisfactorily, so we have
not put safeguards in the gradient.

\begin{knitrout}\scriptsize
\definecolor{shadecolor}{rgb}{0.969, 0.969, 0.969}\color{fgcolor}\begin{kframe}
\begin{alltt}
\hlstd{nll} \hlkwb{<-} \hlkwa{function}\hlstd{(}\hlkwc{y}\hlstd{) \{}
  \hlkwa{if} \hlstd{((}\hlkwd{any}\hlstd{(y} \hlopt{<=} \hlnum{10}\hlopt{*}\hlstd{.Machine}\hlopt{$}\hlstd{double.xmin))} \hlopt{||} \hlstd{(}\hlkwd{sum}\hlstd{(y)}\hlopt{>}\hlnum{1}\hlopt{-}\hlstd{.Machine}\hlopt{$}\hlstd{double.eps))}
         \hlstd{.Machine}\hlopt{$}\hlstd{double.xmax}
  \hlkwa{else}   \hlopt{-} \hlkwd{sum}\hlstd{(}\hlkwd{log}\hlstd{(y))} \hlopt{-} \hlkwd{log}\hlstd{(}\hlnum{1}\hlopt{-}\hlkwd{sum}\hlstd{(y))}
\hlstd{\}}
\hlstd{nll.g} \hlkwb{<-} \hlkwa{function}\hlstd{(}\hlkwc{y}\hlstd{) \{} \hlopt{-} \hlnum{1}\hlopt{/}\hlstd{y} \hlopt{+} \hlnum{1}\hlopt{/}\hlstd{(}\hlnum{1}\hlopt{-}\hlkwd{sum}\hlstd{(y))\}} \hlcom{# so far not safeguarded}
\end{alltt}
\end{kframe}
\end{knitrout}

We can easily try several optimization methods using the \pkg{optimx} package. Here are the
calls, which overall did not perform as well as we would like. Note that we do not ask for
\code{method="ALL"} as we found that some of the methods, in particular those using Powell's
quadratic approximation methods, seem to get "stuck". 

\begin{knitrout}\scriptsize
\definecolor{shadecolor}{rgb}{0.969, 0.969, 0.969}\color{fgcolor}\begin{kframe}
\begin{alltt}
\hlkwd{require}\hlstd{(optimrx,} \hlkwc{quietly}\hlstd{=}\hlnum{TRUE}\hlstd{)}
\hlstd{n}\hlkwb{<-}\hlnum{5}
\hlstd{mset}\hlkwb{<-}\hlkwd{c}\hlstd{(}\hlstr{"L-BFGS-B"}\hlstd{,} \hlstr{"BFGS"}\hlstd{,} \hlstr{"CG"}\hlstd{,} \hlstr{"spg"}\hlstd{,} \hlstr{"ucminf"}\hlstd{,} \hlstr{"nlm"}\hlstd{,} \hlstr{"nlminb"}\hlstd{,} \hlstr{"Rvmmin"}\hlstd{,} \hlstr{"Rcgmin"}\hlstd{)}
\hlstd{a5}\hlkwb{<-}\hlkwd{opm}\hlstd{(}\hlnum{2}\hlopt{:}\hlstd{n}\hlopt{/}\hlstd{n}\hlopt{^}\hlnum{2}\hlstd{, nll,} \hlkwc{gr}\hlstd{=}\hlstr{"grfwd"}\hlstd{,} \hlkwc{method}\hlstd{=mset,} \hlkwc{control}\hlstd{=}\hlkwd{list}\hlstd{(}\hlkwc{dowarn}\hlstd{=}\hlnum{FALSE}\hlstd{))}
\end{alltt}


{\ttfamily\noindent\color{warningcolor}{\#\# Warning in Rvmminu(par = spar, fn = efn, gr = egr, control = mcontrol, ...): Too many gradient evaluations}}\begin{alltt}
\hlstd{a5g}\hlkwb{<-}\hlkwd{opm}\hlstd{(}\hlnum{2}\hlopt{:}\hlstd{n}\hlopt{/}\hlstd{n}\hlopt{^}\hlnum{2}\hlstd{, nll, nll.g,} \hlkwc{method}\hlstd{=mset,} \hlkwc{control}\hlstd{=}\hlkwd{list}\hlstd{(}\hlkwc{dowarn}\hlstd{=}\hlnum{FALSE}\hlstd{))}
\end{alltt}


{\ttfamily\noindent\color{warningcolor}{\#\# Warning in Rvmminu(par = spar, fn = efn, gr = egr, control = mcontrol, ...): Too many gradient evaluations}}\begin{alltt}
\hlstd{a5gb}\hlkwb{<-}\hlkwd{opm}\hlstd{(}\hlnum{2}\hlopt{:}\hlstd{n}\hlopt{/}\hlstd{n}\hlopt{^}\hlnum{2}\hlstd{, nll, nll.g,} \hlkwc{lower}\hlstd{=}\hlnum{0}\hlstd{,} \hlkwc{upper}\hlstd{=}\hlnum{1}\hlstd{,} \hlkwc{method}\hlstd{=mset,} \hlkwc{control}\hlstd{=}\hlkwd{list}\hlstd{(}\hlkwc{dowarn}\hlstd{=}\hlnum{FALSE}\hlstd{))}
\end{alltt}


{\ttfamily\noindent\color{warningcolor}{\#\# Warning in optimr(par, fn, gr, method = meth, lower = lower, upper = upper, : optimr: optim() with bounds ONLY uses L-BFGS-B}}

{\ttfamily\noindent\color{warningcolor}{\#\# Warning in optimr(par, fn, gr, method = meth, lower = lower, upper = upper, : optimr: optim() with bounds ONLY uses L-BFGS-B}}\begin{alltt}
\hlcom{#- a5x <- opm(2:n/n^2, nll, nll.g, method="ALL", control=list(dowarn=FALSE))}
\hlkwd{summary}\hlstd{(a5,}\hlkwc{order}\hlstd{=value)}
\end{alltt}
\begin{verbatim}
##                 p1        p2        p3        p4         value fevals
## Rcgmin   0.2000000 0.2000000 0.2000000 0.2000000  8.047190e+00     51
## spg      0.2000000 0.2000000 0.2000000 0.2000000  8.047190e+00     17
## ucminf   0.2000000 0.2000000 0.2000000 0.2000000  8.047190e+00     14
## Rvmmin   0.2000000 0.2000000 0.2000000 0.2000000  8.047190e+00     28
## CG       0.2000000 0.2000000 0.2000000 0.2000001  8.047190e+00     59
## nlm      0.2000005 0.1999995 0.2000000 0.2000000  8.047190e+00     NA
## BFGS     0.2000007 0.1999989 0.2000011 0.1999981  8.047190e+00     33
## nlminb   0.2000004 0.1999990 0.1999989 0.1999992  8.047190e+00     23
## L-BFGS-B        NA        NA        NA        NA 8.988466e+307     NA
##          gevals convergence kkt1 kkt2 xtime
## Rcgmin       18           0 TRUE TRUE 0.003
## spg          13           0 TRUE TRUE 0.042
## ucminf       14           0 TRUE TRUE 0.002
## Rvmmin     1001           1 TRUE TRUE 0.091
## CG           21           0 TRUE TRUE 0.002
## nlm          11           0 TRUE TRUE 0.001
## BFGS          9           0 TRUE TRUE 0.001
## nlminb       12           0 TRUE TRUE 0.002
## L-BFGS-B     NA        9999   NA   NA 0.002
\end{verbatim}
\begin{alltt}
\hlkwd{summary}\hlstd{(a5g,}\hlkwc{order}\hlstd{=value)}
\end{alltt}
\begin{verbatim}
##                 p1        p2        p3        p4         value fevals
## ucminf   0.2000000 0.2000000 0.2000000 0.2000000  8.047190e+00     14
## spg      0.2000000 0.2000000 0.2000000 0.2000000  8.047190e+00     17
## Rcgmin   0.2000000 0.2000000 0.2000000 0.2000000  8.047190e+00     28
## Rvmmin   0.2000000 0.2000000 0.2000000 0.2000000  8.047190e+00     32
## CG       0.2000000 0.2000000 0.2000000 0.2000000  8.047190e+00     59
## nlm      0.2000006 0.1999995 0.2000000 0.2000000  8.047190e+00     NA
## BFGS     0.2000007 0.1999989 0.2000012 0.1999981  8.047190e+00     33
## nlminb   0.2000004 0.1999990 0.1999989 0.1999992  8.047190e+00     23
## L-BFGS-B        NA        NA        NA        NA 8.988466e+307     NA
##          gevals convergence kkt1 kkt2 xtime
## ucminf       14           0 TRUE TRUE 0.000
## spg          13           0 TRUE TRUE 0.041
## Rcgmin       12           0 TRUE TRUE 0.001
## Rvmmin     1001           1 TRUE TRUE 0.040
## CG           21           0 TRUE TRUE 0.001
## nlm          11           0 TRUE TRUE 0.001
## BFGS          9           0 TRUE TRUE 0.000
## nlminb       12           0 TRUE TRUE 0.001
## L-BFGS-B     NA        9999   NA   NA 0.000
\end{verbatim}
\begin{alltt}
\hlkwd{summary}\hlstd{(a5gb,}\hlkwc{order}\hlstd{=value)}
\end{alltt}
\begin{verbatim}
##                 p1       p2        p3        p4         value fevals
## Rvmmin   0.2000000 0.200000 0.2000000 0.2000000  8.047190e+00     38
## Rcgmin   0.2000000 0.200000 0.2000000 0.2000000  8.047190e+00     18
## spg      0.2000000 0.200000 0.2000000 0.2000000  8.047190e+00     18
## nlminb   0.2000004 0.199999 0.1999989 0.1999992  8.047190e+00     23
## L-BFGS-B        NA       NA        NA        NA 8.988466e+307     NA
## BFGS            NA       NA        NA        NA 8.988466e+307     NA
## CG              NA       NA        NA        NA 8.988466e+307     NA
## ucminf          NA       NA        NA        NA 8.988466e+307     NA
## nlm             NA       NA        NA        NA 8.988466e+307     NA
##          gevals convergence kkt1 kkt2 xtime
## Rvmmin       14           0 TRUE TRUE 0.004
## Rcgmin       10           0 TRUE TRUE 0.002
## spg          13           0 TRUE TRUE 0.043
## nlminb       12           0 TRUE TRUE 0.001
## L-BFGS-B     NA        9999   NA   NA 0.001
## BFGS         NA        9999   NA   NA 0.001
## CG           NA        9999   NA   NA 0.001
## ucminf       NA        9999   NA   NA 0.000
## nlm          NA        9999   NA   NA 0.000
\end{verbatim}
\begin{alltt}
\hlcom{#- summary(a5x,order=value)}
\end{alltt}
\end{kframe}
\end{knitrout}

Most, but not all, of the methods find the solution for the $n=5$ case. The exception (L-BFGS-B) is 
due to the optimization method trying to compute the gradient where sum(x) is greater than 1. We 
have not tried to determine the source of this particular issue. However, it is almost certainly 
a consequence of too large a step. The particular form of $log(1-sum(x))$ 
is undefined once the argument of
the logarithm is negative. Indeed, this is the basis of 
logarithmic barrier functions for constraints. There
is a similar issue with the $n-1$ parameters near zero. Negative values will cause difficulties. 

Numerical gradient approximations will similarly fail, 
particularly as step sizes are often of the order
of 1E-7 in size. There is generally no special check within numerical 
gradient routines to apply bounds. 
Note also that a lower bound of 0 on parameters is not adequate, 
since $log(0)$ is undefined. Choosing a
bound large enough to avoid the logarithm of a zero or negative argument 
while still being small enough 
to allow for parameter optimization is non-trivial.



\section{Transformed problems or parameters}

When problems give difficulties, it is common to re-formulate them by transformations of the function
or the parameters. 

\subsection{Using a projection}

Objective functions defined by $(-1)*\prod{\B{x}}$ or $(-1)*\sum{log(\B{x})}$ will change 
with the scale of the parameters. Moreover, the constraint $\sum{\B{x}}=1$ 
effectively imposes the scaling $\B{x_{scaled}} = \B{x}/\sum{\B{x}}$. The 
optimizer \code{spg} from package \pkg{BB} allows us to project our search 
direction to satisfy constraints. Thus, we could use the following approach. Thanks
to Ravi Varadhan for the suggestion.

\begin{knitrout}\scriptsize
\definecolor{shadecolor}{rgb}{0.969, 0.969, 0.969}\color{fgcolor}\begin{kframe}
\begin{alltt}
\hlkwd{require}\hlstd{(BB,} \hlkwc{quietly}\hlstd{=}\hlnum{TRUE}\hlstd{)}
\hlstd{nllrv} \hlkwb{<-} \hlkwa{function}\hlstd{(}\hlkwc{x}\hlstd{) \{}\hlopt{-} \hlkwd{sum}\hlstd{(}\hlkwd{log}\hlstd{(x))\}}
\hlstd{nllrv.g} \hlkwb{<-} \hlkwa{function}\hlstd{(}\hlkwc{x}\hlstd{) \{}\hlopt{-} \hlnum{1}\hlopt{/}\hlstd{x \}}
\hlstd{proj} \hlkwb{<-} \hlkwa{function}\hlstd{(}\hlkwc{x}\hlstd{) \{x}\hlopt{/}\hlkwd{sum}\hlstd{(x)\}}
\hlstd{n} \hlkwb{<-} \hlnum{5}
\hlstd{tspg}\hlkwb{<-}\hlkwd{system.time}\hlstd{(aspg} \hlkwb{<-} \hlkwd{spg}\hlstd{(}\hlkwc{par}\hlstd{=(}\hlnum{1}\hlopt{:}\hlstd{n)}\hlopt{/}\hlstd{n}\hlopt{^}\hlnum{2}\hlstd{,} \hlkwc{fn}\hlstd{=nllrv,} \hlkwc{gr}\hlstd{=nllrv.g,} \hlkwc{project}\hlstd{=proj))[[}\hlnum{3}\hlstd{]]}
\end{alltt}
\begin{verbatim}
## iter:  0  f-value:  11.30689  pgrad:  0.3607565
\end{verbatim}
\begin{alltt}
\hlstd{tspgn}\hlkwb{<-}\hlkwd{system.time}\hlstd{(aspgn} \hlkwb{<-} \hlkwd{spg}\hlstd{(}\hlkwc{par}\hlstd{=(}\hlnum{1}\hlopt{:}\hlstd{n)}\hlopt{/}\hlstd{n}\hlopt{^}\hlnum{2}\hlstd{,} \hlkwc{fn}\hlstd{=nllrv,} \hlkwc{project}\hlstd{=proj))[[}\hlnum{3}\hlstd{]]}
\end{alltt}
\begin{verbatim}
## iter:  0  f-value:  11.30689  pgrad:  0.1333334
\end{verbatim}
\begin{alltt}
\hlkwd{cat}\hlstd{(}\hlstr{"Times: with gradient ="}\hlstd{,tspg,}\hlstr{"   using numerical approx.="}\hlstd{, tspgn,}\hlstr{"\textbackslash{}n"}\hlstd{)}
\end{alltt}
\begin{verbatim}
## Times: with gradient = 0.044    using numerical approx.= 0.042
\end{verbatim}
\begin{alltt}
\hlkwd{cat}\hlstd{(}\hlstr{"F_optimal: with gradient="}\hlstd{,aspg}\hlopt{$}\hlstd{value,}\hlstr{"  num. approx.="}\hlstd{,aspgn}\hlopt{$}\hlstd{value,}\hlstr{"\textbackslash{}n"}\hlstd{)}
\end{alltt}
\begin{verbatim}
## F_optimal: with gradient= 8.04719   num. approx.= 8.04719
\end{verbatim}
\begin{alltt}
\hlstd{pbest}\hlkwb{<-}\hlkwd{rep}\hlstd{(}\hlnum{1}\hlopt{/}\hlstd{n, n)}
\hlkwd{cat}\hlstd{(}\hlstr{"fbest = "}\hlstd{,}\hlkwd{nllrv}\hlstd{(pbest),}\hlstr{"  when all parameters = "}\hlstd{, pbest[}\hlnum{1}\hlstd{],}\hlstr{"\textbackslash{}n"}\hlstd{)}
\end{alltt}
\begin{verbatim}
## fbest =  8.04719   when all parameters =  0.2
\end{verbatim}
\begin{alltt}
\hlkwd{cat}\hlstd{(}\hlstr{"deviations:  with gradient="}\hlstd{,}\hlkwd{max}\hlstd{(}\hlkwd{abs}\hlstd{(aspg}\hlopt{$}\hlstd{par}\hlopt{-}\hlstd{pbest)),}\hlstr{"   num. approx.="}\hlstd{,}\hlkwd{max}\hlstd{(}\hlkwd{abs}\hlstd{(aspg}\hlopt{$}\hlstd{par}\hlopt{-}\hlstd{pbest)),}\hlstr{"\textbackslash{}n"}\hlstd{)}
\end{alltt}
\begin{verbatim}
## deviations:  with gradient= 3.81244e-06    num. approx.= 3.81244e-06
\end{verbatim}
\end{kframe}
\end{knitrout}

Here the projection \code{proj} is the key to success of method 
\code{spg}. Other methods do not have the flexibility to impose the projection directly. 
We would need to carefully build the projection into
the function(s) and/or the method codes. 
This was done by \cite{Geradin71} for the Rayleigh quotient 
problem, but requires a number of changes to the program code.

\subsection{$log()$ transformation of parameters}

A common method to ensure parameters are positive is to transform 
them. In the present case, optimizing over
parameters that are the logarithms of the parameters above 
ensures we have positive arguments to most of the
elements of the negative log likelihood. Here is the code. 
Note that the parameters used in optimization
are "lx" and not x.

\begin{knitrout}\scriptsize
\definecolor{shadecolor}{rgb}{0.969, 0.969, 0.969}\color{fgcolor}\begin{kframe}
\begin{alltt}
\hlstd{enll} \hlkwb{<-} \hlkwa{function}\hlstd{(}\hlkwc{lx}\hlstd{) \{}
    \hlstd{x}\hlkwb{<-}\hlkwd{exp}\hlstd{(lx)}
    \hlstd{fval}\hlkwb{<-}  \hlopt{-} \hlkwd{sum}\hlstd{(} \hlkwd{log}\hlstd{( x}\hlopt{/}\hlkwd{sum}\hlstd{(x) ) )}
\hlstd{\}}
\hlstd{enll.g} \hlkwb{<-} \hlkwa{function}\hlstd{(}\hlkwc{lx}\hlstd{)\{}
    \hlstd{x}\hlkwb{<-}\hlkwd{exp}\hlstd{(lx)}
    \hlstd{g}\hlkwb{<-}\hlkwd{length}\hlstd{(x)}\hlopt{/}\hlkwd{sum}\hlstd{(x)} \hlopt{-} \hlnum{1}\hlopt{/}\hlstd{x}
    \hlstd{gval}\hlkwb{<-}\hlstd{g}\hlopt{*}\hlkwd{exp}\hlstd{(lx)}
\hlstd{\}}
\end{alltt}
\end{kframe}
\end{knitrout}

But where is our constraint? Here we have noted that we could define the objective 
function only to within the scaling  $\B{x}/\sum(\B{x})$. There is a minor 
nuisance, in that we need to re-scale our 
parameters after solution to have them in a standard form. 
This is most noticeable if one uses \pkg{optimx}
and displays the results of \code{all.methods}. In the following, we
extract the best solution for the 5-parameter problem.

\begin{knitrout}\scriptsize
\definecolor{shadecolor}{rgb}{0.969, 0.969, 0.969}\color{fgcolor}\begin{kframe}
\begin{alltt}
\hlkwd{require}\hlstd{(optimrx,} \hlkwc{quietly}\hlstd{=}\hlnum{TRUE}\hlstd{)} \hlcom{# just to be sure}
\hlstd{st}\hlkwb{<-}\hlnum{1}\hlopt{:}\hlnum{5}\hlopt{/}\hlnum{10} \hlcom{# 5 parameters, crude scaling to start}
\hlstd{a5x}\hlkwb{<-}\hlkwd{opm}\hlstd{(st, enll, enll.g,} \hlkwc{method}\hlstd{=}\hlstr{"ALL"}\hlstd{,} \hlkwc{control}\hlstd{=}\hlkwd{list}\hlstd{(}\hlkwc{trace}\hlstd{=}\hlnum{0}\hlstd{))}
\hlstd{a5xbyvalue}\hlkwb{<-}\hlkwd{summary}\hlstd{(a5x,} \hlkwc{order}\hlstd{=value)}
\hlstd{xnor}\hlkwb{<-}\hlstd{a5xbyvalue[}\hlnum{1}\hlstd{,} \hlnum{1}\hlopt{:}\hlnum{5}\hlstd{]} \hlcom{# get the 5 parameters of "best" solution}
\hlstd{xnor}\hlkwb{<-}\hlstd{xnor}\hlopt{/}\hlkwd{sum}\hlstd{(xnor)}
\hlkwd{cat}\hlstd{(}\hlstr{"normalized parameters:"}\hlstd{)}
\end{alltt}
\begin{verbatim}
## normalized parameters:
\end{verbatim}
\begin{alltt}
\hlkwd{print}\hlstd{(xnor)}
\end{alltt}
\begin{verbatim}
##       p1  p2  p3  p4  p5
## BFGS 0.2 0.2 0.2 0.2 0.2
\end{verbatim}
\end{kframe}
\end{knitrout}


While there are reasons to think that the indeterminacy
might upset the optimization codes, in practice, the objective 
and gradient above are generally
well-behaved, though they did reveal that tests of the size 
of the gradient used, in particular, to
decide to terminate iterations in \pkg{Rcgmin} were too 
hasty in stopping progress for problems
with larger numbers of parameters. A user-specified tolerance is now allowed; for
example \code{control=list(tol=1e-12)}. 

Let us try a larger problem in 100 parameters.

\begin{knitrout}\scriptsize
\definecolor{shadecolor}{rgb}{0.969, 0.969, 0.969}\color{fgcolor}\begin{kframe}
\begin{alltt}
\hlkwd{require}\hlstd{(Rcgmin,} \hlkwc{quietly}\hlstd{=}\hlnum{TRUE}\hlstd{)}
\hlstd{st}\hlkwb{<-}\hlnum{1}\hlopt{:}\hlnum{100}\hlopt{/}\hlnum{1e3} \hlcom{# large}
\hlstd{stenll}\hlkwb{<-}\hlkwd{enll}\hlstd{(st)}
\hlkwd{cat}\hlstd{(}\hlstr{"Initial function value ="}\hlstd{,stenll,}\hlstr{"\textbackslash{}n"}\hlstd{)}
\end{alltt}
\begin{verbatim}
## Initial function value = 460.5587
\end{verbatim}
\begin{alltt}
\hlstd{tym}\hlkwb{<-}\hlkwd{system.time}\hlstd{(acgbig}\hlkwb{<-}\hlkwd{Rcgmin}\hlstd{(st, enll, enll.g,} \hlkwc{control}\hlstd{=}\hlkwd{list}\hlstd{(}\hlkwc{trace}\hlstd{=}\hlnum{0}\hlstd{,} \hlkwc{tol}\hlstd{=}\hlnum{1e-32}\hlstd{)))[[}\hlnum{3}\hlstd{]]}
\hlkwd{cat}\hlstd{(}\hlstr{"Time = "}\hlstd{,tym,}\hlstr{"  fval="}\hlstd{,acgbig}\hlopt{$}\hlstd{value,}\hlstr{"\textbackslash{}n"}\hlstd{)}
\end{alltt}
\begin{verbatim}
## Time =  0.081   fval= 460.517
\end{verbatim}
\begin{alltt}
\hlstd{xnor}\hlkwb{<-}\hlstd{acgbig}\hlopt{$}\hlstd{par}\hlopt{/}\hlkwd{sum}\hlstd{(acgbig}\hlopt{$}\hlstd{par)}
\hlkwd{print}\hlstd{(xnor)}
\end{alltt}
\begin{verbatim}
##   [1] 0.01 0.01 0.01 0.01 0.01 0.01 0.01 0.01 0.01 0.01 0.01 0.01 0.01 0.01
##  [15] 0.01 0.01 0.01 0.01 0.01 0.01 0.01 0.01 0.01 0.01 0.01 0.01 0.01 0.01
##  [29] 0.01 0.01 0.01 0.01 0.01 0.01 0.01 0.01 0.01 0.01 0.01 0.01 0.01 0.01
##  [43] 0.01 0.01 0.01 0.01 0.01 0.01 0.01 0.01 0.01 0.01 0.01 0.01 0.01 0.01
##  [57] 0.01 0.01 0.01 0.01 0.01 0.01 0.01 0.01 0.01 0.01 0.01 0.01 0.01 0.01
##  [71] 0.01 0.01 0.01 0.01 0.01 0.01 0.01 0.01 0.01 0.01 0.01 0.01 0.01 0.01
##  [85] 0.01 0.01 0.01 0.01 0.01 0.01 0.01 0.01 0.01 0.01 0.01 0.01 0.01 0.01
##  [99] 0.01 0.01
\end{verbatim}
\end{kframe}
\end{knitrout}

One worrying aspect of the solution is that the objective function 
at the start and end differ by a tiny amount. 

\subsection{Another transformation}

A slightly different transformation or projection is inspired by spherical coordinates.




\begin{knitrout}\scriptsize
\definecolor{shadecolor}{rgb}{0.969, 0.969, 0.969}\color{fgcolor}\begin{kframe}
\begin{alltt}
\hlstd{proj2} \hlkwb{<-} \hlkwa{function}\hlstd{(}\hlkwc{theta}\hlstd{) \{}
    \hlstd{theta2} \hlkwb{<-} \hlstd{theta}\hlopt{^}\hlnum{2}
    \hlstd{s2} \hlkwb{<-} \hlstd{theta2} \hlopt{/} \hlstd{(}\hlnum{1} \hlopt{+} \hlstd{theta2)}
    \hlkwd{cumprod}\hlstd{(}\hlkwd{c}\hlstd{(}\hlnum{1}\hlstd{, s2))} \hlopt{*} \hlkwd{c}\hlstd{(}\hlnum{1}\hlopt{-}\hlstd{s2,} \hlnum{1}\hlstd{)}
 \hlstd{\}}
\hlstd{obj} \hlkwb{<-} \hlkwa{function}\hlstd{(}\hlkwc{theta}\hlstd{)} \hlopt{-} \hlkwd{sum}\hlstd{(}\hlkwd{log}\hlstd{(}\hlkwd{proj2}\hlstd{(theta)))}
 \hlstd{n} \hlkwb{<-} \hlnum{5}
 \hlstd{ans} \hlkwb{<-} \hlkwd{spg}\hlstd{(}\hlkwd{seq}\hlstd{(n}\hlopt{-}\hlnum{1}\hlstd{), obj)}
\end{alltt}
\begin{verbatim}
## iter:  0  f-value:  11.15175  pgrad:  3 
## iter:  10  f-value:  8.78015  pgrad:  0.5806909 
## iter:  20  f-value:  8.04719  pgrad:  3.925749e-06
\end{verbatim}
\begin{alltt}
 \hlkwd{proj2}\hlstd{(ans}\hlopt{$}\hlstd{par)}
\end{alltt}
\begin{verbatim}
## [1] 0.2000000 0.2000007 0.2000002 0.1999996 0.1999995
\end{verbatim}
\end{kframe}
\end{knitrout}


\begin{knitrout}\scriptsize
\definecolor{shadecolor}{rgb}{0.969, 0.969, 0.969}\color{fgcolor}\begin{kframe}
\begin{alltt}
\hlstd{n}\hlkwb{<-}\hlnum{100}
\hlstd{ans100} \hlkwb{<-} \hlkwd{spg}\hlstd{(}\hlkwd{seq}\hlstd{(n}\hlopt{-}\hlnum{1}\hlstd{), obj,} \hlkwc{control}\hlstd{=}\hlkwd{list}\hlstd{(}\hlkwc{trace}\hlstd{=}\hlnum{FALSE}\hlstd{),} \hlkwc{quiet}\hlstd{=}\hlnum{TRUE}\hlstd{)}
\hlkwd{proj2}\hlstd{( (ans100}\hlopt{$}\hlstd{par) )}
\end{alltt}
\begin{verbatim}
##   [1] 0.009999999 0.010000001 0.010000000 0.010000000 0.009999999
##   [6] 0.010000003 0.010000000 0.010000001 0.010000000 0.010000000
##  [11] 0.010000001 0.010000000 0.010000000 0.010000002 0.010000002
##  [16] 0.010000001 0.009999999 0.010000002 0.010000000 0.010000002
##  [21] 0.010000000 0.010000000 0.010000002 0.010000001 0.010000001
##  [26] 0.010000001 0.010000000 0.010000002 0.010000001 0.010000001
##  [31] 0.010000000 0.010000001 0.010000002 0.010000002 0.010000001
##  [36] 0.010000002 0.010000001 0.010000000 0.010000001 0.010000001
##  [41] 0.009999999 0.009999996 0.010000002 0.010000002 0.010000002
##  [46] 0.009999996 0.009999997 0.009999999 0.010000000 0.009999999
##  [51] 0.009999998 0.010000002 0.010000000 0.009999999 0.010000000
##  [56] 0.009999996 0.010000002 0.010000002 0.009999998 0.010000000
##  [61] 0.010000000 0.009999999 0.010000000 0.010000000 0.010000001
##  [66] 0.009999999 0.010000000 0.010000000 0.010000000 0.010000000
##  [71] 0.009999999 0.010000000 0.009999999 0.010000000 0.009999999
##  [76] 0.010000000 0.010000000 0.009999999 0.010000000 0.009999999
##  [81] 0.009999999 0.009999999 0.010000000 0.010000001 0.009999999
##  [86] 0.009999999 0.010000000 0.010000001 0.010000000 0.009999998
##  [91] 0.009999998 0.009999999 0.010000000 0.010000000 0.010000000
##  [96] 0.009999997 0.010000038 0.009999935 0.010000003 0.010000023
\end{verbatim}
\end{kframe}
\end{knitrout}

Since this transformation is embedded into the objective function, we could run all the
optimizers in \pkg{optimx} as follows. This takes some time, as the derivative-free
methods appear to have more difficulty with this formulation. Moreover, \code{Rcgmin}
and \code{Rvmmin} are not recommended when an analytic gradient is not provided.

\begin{knitrout}\scriptsize
\definecolor{shadecolor}{rgb}{0.969, 0.969, 0.969}\color{fgcolor}\begin{kframe}
\begin{alltt}
\hlstd{allans}\hlkwb{<-} \hlkwd{opm}\hlstd{(}\hlkwd{seq}\hlstd{(n}\hlopt{-}\hlnum{1}\hlstd{), obj,} \hlkwc{gr}\hlstd{=}\hlstr{"grfwd"}\hlstd{,} \hlkwc{method}\hlstd{=}\hlstr{"ALL"}\hlstd{,} \hlkwc{control}\hlstd{=}\hlkwd{list}\hlstd{(}\hlkwc{dowarn}\hlstd{=}\hlnum{FALSE}\hlstd{))}
\end{alltt}


{\ttfamily\noindent\color{warningcolor}{\#\# Warning in Rvmminu(par = spar, fn = efn, gr = egr, control = mcontrol, ...): Too many gradient evaluations}}

{\ttfamily\noindent\color{warningcolor}{\#\# Warning in commonArgs(par + 0, fn, control, environment()): maxfun < 10 * length(par)\textasciicircum{}2 is not recommended.}}

{\ttfamily\noindent\color{warningcolor}{\#\# Warning in commonArgs(par, fn, control, environment()): maxfun < 10 * length(par)\textasciicircum{}2 is not recommended.}}

{\ttfamily\noindent\color{warningcolor}{\#\# Warning in nmk(par = spar, fn = efn, control = mcontrol, ...): Nelder-Mead should not be used for high-dimensional optimization}}

{\ttfamily\noindent\color{warningcolor}{\#\# Warning in optimr(par, fn, gr, method = meth, lower = lower, upper = upper, : Maximum number of fevals exceeded\ \ Restarts for stagnation =0}}\begin{alltt}
\hlkwd{summary}\hlstd{(allans,} \hlkwc{order} \hlstd{=} \hlstr{"list(round(value, 3), fevals)"}\hlstd{,} \hlkwc{par.select} \hlstd{=} \hlnum{FALSE}\hlstd{)}
\end{alltt}
\begin{verbatim}
##                     value fevals gevals convergence  kkt1  kkt2 xtime
## spg          4.605170e+02    230    212           0  TRUE  TRUE 0.440
## lbfgsb3      4.605170e+02    267    267           0  TRUE  TRUE 0.662
## L-BFGS-B     4.605170e+02    280    280           0  TRUE  TRUE 0.540
## ucminf       4.605170e+02    310    310           0  TRUE  TRUE 0.556
## BFGS         4.605170e+02    312    217           0  TRUE  TRUE 0.401
## Rvmmin       4.605170e+02    353   5001           1  TRUE  TRUE 9.748
## Rcgmin       4.605170e+02   1612   1023           0  TRUE  TRUE 1.800
## CG           4.605170e+02   4039   1723           0  TRUE  TRUE 3.132
## hjkb         4.605170e+02  20414     NA           0  TRUE  TRUE 0.417
## bobyqa       4.605170e+02  21493     NA           0  TRUE  TRUE 7.635
## hjn          4.605170e+02  23292     NA           0  TRUE  TRUE 0.583
## nlm          4.605170e+02     NA    203           0  TRUE  TRUE 0.661
## Rtnmin       4.756232e+02    926    926           3  TRUE FALSE 2.579
## nlminb       4.822372e+02    189    151           1 FALSE FALSE 0.302
## nmkb         7.202616e+02   5045     NA           1  TRUE FALSE 2.548
## lbfgs        7.326367e+02     NA     NA       -1001  TRUE FALSE 0.026
## Nelder-Mead  7.439745e+02   5002     NA           1  TRUE FALSE 0.115
## newuoa      8.988466e+307     NA     NA        9999    NA    NA 0.003
\end{verbatim}
\end{kframe}
\end{knitrout}



\subsection{Use the gradient equations}

Another approach is to "solve" the gradient equations. We can do this with 
a sum of squares minimizer, though the \code{nls} function in \R\ is 
specifically NOT useful as it cannot deal
with small or zero residuals. However, \code{nlfb} 
from package \pkg{nlmrt} is capable of dealing
with such problems. Unfortunately, it will be slow as it has to 
generate the Jacobian by numerical
approximation unless we can provide a function to prepare the 
Jacobian analytically. Moreover,
the determination of the Jacobian is still subject to 
the unfortunate scaling issues we have
been confronting throughout this article.


\section{The Rayleigh Quotient}

The maximal and minimal eigensolutions of a symmetric matrix $A$ are extrema of the Rayleigh Quotient

$ R(x) =  (x' A x)  / (x' x) $

We can also deal with generalized eigenproblems of the form 

$A x = e B x$

where B is symmetric and positive definite by using the Rayleigh Quotient

$ R_g(x) =  (x' A x)  / (x' B x) $


Once again, the objective is scaled by the parameters, this time by their 
sum of squares. Alternatively, 
we may think of requiring the \B{normalized} eigensolution, which is given as 

$ x_{normalized} = x/sqrt(x' x) $

We will first try the projected gradient method \code{spg} from \pkg{BB}. 
Below is the code, where our test uses
a matrix called the Moler matrix \cite[Appendix 1]{cnm79}. We caution that there
are faster ways to compute this matrix in \R\ \citep{RQtimes12} where different
approaches to speed up \R\ computations are discussed. Here we are concerned 
with getting the solutions correctly rather than the speed of so doing. Note
that to get the solution with the most-positive eigenvalue, we minimize the
Rayleigh quotient of the matrix multiplied by -1. This is solution \code{tmax}.

\begin{knitrout}\scriptsize
\definecolor{shadecolor}{rgb}{0.969, 0.969, 0.969}\color{fgcolor}\begin{kframe}
\begin{alltt}
\hlstd{molerbuild}\hlkwb{<-}\hlkwa{function}\hlstd{(}\hlkwc{n}\hlstd{)\{} \hlcom{# Create the moler matrix of order n}
   \hlcom{# A[i,j] = i for i=j, min(i,j)-2 otherwise}
   \hlstd{A} \hlkwb{<-} \hlkwd{matrix}\hlstd{(}\hlnum{0}\hlstd{,} \hlkwc{nrow} \hlstd{= n,} \hlkwc{ncol} \hlstd{= n)}
   \hlstd{j} \hlkwb{<-} \hlnum{1}\hlopt{:}\hlstd{n}
   \hlkwa{for} \hlstd{(i} \hlkwa{in} \hlnum{1}\hlopt{:}\hlstd{n) \{}
      \hlstd{A[i,} \hlnum{1}\hlopt{:}\hlstd{i]} \hlkwb{<-} \hlkwd{pmin}\hlstd{(i,} \hlnum{1}\hlopt{:}\hlstd{i)} \hlopt{-} \hlnum{2}
   \hlstd{\}}
   \hlstd{A} \hlkwb{<-} \hlstd{A} \hlopt{+} \hlkwd{t}\hlstd{(A)}
   \hlkwd{diag}\hlstd{(A)} \hlkwb{<-} \hlnum{1}\hlopt{:}\hlstd{n}
   \hlstd{A}
\hlstd{\}}

\hlstd{raynum}\hlkwb{<-}\hlkwa{function}\hlstd{(}\hlkwc{x}\hlstd{,} \hlkwc{A}\hlstd{)\{}
   \hlstd{rayquo}\hlkwb{<-}\hlkwd{as.numeric}\hlstd{((}\hlkwd{t}\hlstd{(x)}\hlopt\hlstd{A)}\hlopt\hlstd{x)}
\hlstd{\}}

\hlstd{proj}\hlkwb{<-}\hlkwa{function}\hlstd{(}\hlkwc{x}\hlstd{) \{ x}\hlopt{/}\hlkwd{sqrt}\hlstd{(}\hlkwd{crossprod}\hlstd{(x)) \}}

\hlkwd{require}\hlstd{(BB,} \hlkwc{quietly}\hlstd{=}\hlnum{TRUE}\hlstd{)}
\hlstd{n}\hlkwb{<-}\hlnum{10}
\hlstd{x}\hlkwb{<-}\hlkwd{rep}\hlstd{(}\hlnum{1}\hlstd{,n)}
\hlstd{A}\hlkwb{<-}\hlkwd{molerbuild}\hlstd{(n)}
\hlstd{tmin}\hlkwb{<-}\hlkwd{system.time}\hlstd{(asprqmin}\hlkwb{<-}\hlkwd{spg}\hlstd{(x,} \hlkwc{fn}\hlstd{=raynum,} \hlkwc{project}\hlstd{=proj,} \hlkwc{A}\hlstd{=A))[[}\hlnum{3}\hlstd{]]}
\end{alltt}
\begin{verbatim}
## iter:  0  f-value:  205  pgrad:  3.089431e-09
\end{verbatim}


{\ttfamily\noindent\color{warningcolor}{\#\# Warning in spg(x, fn = raynum, project = proj, A = A): convergence tolerance satisified at intial parameter values.}}\begin{alltt}
\hlstd{tmax}\hlkwb{<-}\hlkwd{system.time}\hlstd{(asprqmax}\hlkwb{<-}\hlkwd{spg}\hlstd{(x,} \hlkwc{fn}\hlstd{=raynum,} \hlkwc{project}\hlstd{=proj,} \hlkwc{A}\hlstd{=}\hlopt{-}\hlstd{A))[[}\hlnum{3}\hlstd{]]}
\end{alltt}
\begin{verbatim}
## iter:  0  f-value:  -205  pgrad:  0.6324555
\end{verbatim}


{\ttfamily\noindent\color{warningcolor}{\#\# Warning in spg(x, fn = raynum, project = proj, A = -A): Unsuccessful convergence.}}\begin{alltt}
\hlkwd{cat}\hlstd{(}\hlstr{"maximal eigensolution: Value="}\hlstd{,asprqmax}\hlopt{$}\hlstd{value,}\hlstr{"in time "}\hlstd{,tmax,}\hlstr{"\textbackslash{}n"}\hlstd{)}
\end{alltt}
\begin{verbatim}
## maximal eigensolution: Value= -205 in time  0.286
\end{verbatim}
\begin{alltt}
\hlkwd{print}\hlstd{(asprqmax}\hlopt{$}\hlstd{par)}
\end{alltt}
\begin{verbatim}
##  [1] 0.3162278 0.3162278 0.3162278 0.3162278 0.3162278 0.3162278 0.3162278
##  [8] 0.3162278 0.3162278 0.3162278
\end{verbatim}
\begin{alltt}
\hlkwd{cat}\hlstd{(}\hlstr{"minimal eigensolution: Value="}\hlstd{,asprqmin}\hlopt{$}\hlstd{value,}\hlstr{"in time "}\hlstd{,tmin,}\hlstr{"\textbackslash{}n"}\hlstd{)}
\end{alltt}
\begin{verbatim}
## minimal eigensolution: Value= 205 in time  0.043
\end{verbatim}
\begin{alltt}
\hlkwd{print}\hlstd{(asprqmin}\hlopt{$}\hlstd{par)}
\end{alltt}
\begin{verbatim}
##  [1] 0.3162278 0.3162278 0.3162278 0.3162278 0.3162278 0.3162278 0.3162278
##  [8] 0.3162278 0.3162278 0.3162278
\end{verbatim}
\end{kframe}
\end{knitrout}

For the record, these results compare well with eigenvalues from eigen().

If we ignore the constraint, and simply perform the optimization, we can
get satisfactory solutions, though comparisons require that we normalize 
the parameters post-optimization. We can check if the scale of the eigenvectors 
is becoming large by computing the norm of the final parameter vector. In 
tests on the Moler matrix up to dimension 100, none grew to a worrying size. 

For comparison, we also ran a specialized Geradin routine as implemented in \R\ by one of 
us (JN). This gave equivalent answers, albeit more efficiently. For those interested, the
Geradin routine is available as referenced in \citep{RQtimes12}.


\section{The R-help example}

As a final example, let us use our present techniques to solve the 
problem posed by Lanh Tran on R-help. We will use
only a method that scales the parameters directly inside the objective function and 
not bother with gradients for this small problem. 

\begin{knitrout}\scriptsize
\definecolor{shadecolor}{rgb}{0.969, 0.969, 0.969}\color{fgcolor}\begin{kframe}
\begin{alltt}
\hlstd{ssums}\hlkwb{<-}\hlkwa{function}\hlstd{(}\hlkwc{x}\hlstd{)\{}
  \hlstd{n}\hlkwb{<-}\hlkwd{length}\hlstd{(x)}
  \hlstd{tt}\hlkwb{<-}\hlkwd{sum}\hlstd{(x)}
  \hlstd{ss}\hlkwb{<-}\hlnum{1}\hlopt{:}\hlstd{n}
  \hlstd{xx}\hlkwb{<-}\hlstd{(x}\hlopt{/}\hlstd{tt)}\hlopt{*}\hlstd{(x}\hlopt{/}\hlstd{tt)}
  \hlkwd{sum}\hlstd{(ss}\hlopt{*}\hlstd{xx)}
\hlstd{\}}

\hlkwd{cat}\hlstd{(}\hlstr{"Try penalized sum\textbackslash{}n"}\hlstd{)}
\end{alltt}
\begin{verbatim}
## Try penalized sum
\end{verbatim}
\begin{alltt}
\hlkwd{require}\hlstd{(optimx)}
\hlstd{st}\hlkwb{<-}\hlkwd{runif}\hlstd{(}\hlnum{3}\hlstd{)}
\hlstd{aos}\hlkwb{<-}\hlkwd{opm}\hlstd{(st, ssums,} \hlkwc{gr}\hlstd{=}\hlstr{"grcentral"}\hlstd{,} \hlkwc{method}\hlstd{=}\hlstr{"ALL"}\hlstd{)}
\end{alltt}


{\ttfamily\noindent\color{warningcolor}{\#\# Warning in Rvmminu(par = spar, fn = efn, gr = egr, control = mcontrol, ...): Too many gradient evaluations}}

{\ttfamily\noindent\color{warningcolor}{\#\# Warning in optimr(par, fn, gr, method = meth, lower = lower, upper = upper, : Successful convergence\ \ Restarts for stagnation =0}}\begin{alltt}
\hlcom{# rescale the parameters}
\hlstd{nsol}\hlkwb{<-}\hlkwd{dim}\hlstd{(aos)[}\hlnum{1}\hlstd{]}
\hlkwa{for} \hlstd{(i} \hlkwa{in} \hlnum{1}\hlopt{:}\hlstd{nsol)\{}
  \hlstd{tpar}\hlkwb{<-}\hlstd{aos[i,}\hlnum{1}\hlopt{:}\hlnum{3}\hlstd{]}
  \hlstd{ntpar}\hlkwb{<-}\hlkwd{sum}\hlstd{(tpar)}
  \hlstd{tpar}\hlkwb{<-}\hlstd{tpar}\hlopt{/}\hlstd{ntpar}
\hlcom{#  cat("Method ",aos[i, "meth"]," gives fval =", ssums(tpar))}
  \hlstd{aos[i,} \hlnum{1}\hlopt{:}\hlnum{3}\hlstd{]}\hlkwb{<-}\hlstd{tpar}

\hlstd{\}}

\hlkwd{summary}\hlstd{(aos,}\hlkwc{order}\hlstd{=value)[}\hlnum{1}\hlopt{:}\hlnum{5}\hlstd{,]}
\end{alltt}
\begin{verbatim}
##               p1        p2        p3     value fevals gevals convergence
## newuoa 0.5454545 0.2727273 0.1818182 0.5454545     51     NA           0
## hjn    0.5454545 0.2727273 0.1818182 0.5454545    228     NA           0
## Rvmmin 0.5454545 0.2727273 0.1818182 0.5454545     10   1001           1
## nlminb 0.5454545 0.2727273 0.1818182 0.5454545      9      9           0
## bobyqa 0.5454546 0.2727272 0.1818182 0.5454545     55     NA           0
##        kkt1  kkt2 xtime
## newuoa TRUE FALSE 0.002
## hjn    TRUE FALSE 0.004
## Rvmmin TRUE FALSE 0.092
## nlminb TRUE FALSE 0.000
## bobyqa TRUE FALSE 0.001
\end{verbatim}
\end{kframe}
\end{knitrout}



\begin{knitrout}\scriptsize
\definecolor{shadecolor}{rgb}{0.969, 0.969, 0.969}\color{fgcolor}\begin{kframe}
\begin{alltt}
\hlstd{ssum}\hlkwb{<-}\hlkwa{function}\hlstd{(}\hlkwc{x}\hlstd{)\{}
  \hlstd{n}\hlkwb{<-}\hlkwd{length}\hlstd{(x)}
  \hlstd{ss}\hlkwb{<-}\hlnum{1}\hlopt{:}\hlstd{n}
  \hlstd{xx}\hlkwb{<-}\hlstd{x}\hlopt{*}\hlstd{x}
  \hlkwd{sum}\hlstd{(ss}\hlopt{*}\hlstd{xx)}
\hlstd{\}}
\hlstd{proj.simplex} \hlkwb{<-} \hlkwa{function}\hlstd{(}\hlkwc{y}\hlstd{) \{}
\hlcom{# project an n-dim vector y to the simplex Dn}
\hlcom{# Dn = \{ x : x n-dim, 1 >= x >= 0, sum(x) = 1\}}
\hlcom{# Ravi Varadhan, Johns Hopkins University}
\hlcom{# August 8, 2012}

\hlstd{n} \hlkwb{<-} \hlkwd{length}\hlstd{(y)}
\hlstd{sy} \hlkwb{<-} \hlkwd{sort}\hlstd{(y,} \hlkwc{decreasing}\hlstd{=}\hlnum{TRUE}\hlstd{)}
\hlstd{csy} \hlkwb{<-} \hlkwd{cumsum}\hlstd{(sy)}
\hlstd{rho} \hlkwb{<-} \hlkwd{max}\hlstd{(}\hlkwd{which}\hlstd{(sy} \hlopt{>} \hlstd{(csy} \hlopt{-} \hlnum{1}\hlstd{)}\hlopt{/}\hlstd{(}\hlnum{1}\hlopt{:}\hlstd{n)))}
\hlstd{theta} \hlkwb{<-} \hlstd{(csy[rho]} \hlopt{-} \hlnum{1}\hlstd{)} \hlopt{/} \hlstd{rho}
\hlkwd{return}\hlstd{(}\hlkwd{pmax}\hlstd{(}\hlnum{0}\hlstd{, y} \hlopt{-} \hlstd{theta))}
\hlstd{\}}
\hlstd{as}\hlkwb{<-}\hlkwd{spg}\hlstd{(st, ssum,} \hlkwc{project}\hlstd{=proj.simplex)}
\end{alltt}
\begin{verbatim}
## iter:  0  f-value:  3.246022  pgrad:  0.7459304 
## iter:  10  f-value:  0.5454545  pgrad:  8.601083e-05
\end{verbatim}
\begin{alltt}
\hlkwd{cat}\hlstd{(}\hlstr{"Using project.simplex with spg: fmin="}\hlstd{,as}\hlopt{$}\hlstd{value,}\hlstr{" at \textbackslash{}n"}\hlstd{)}
\end{alltt}
\begin{verbatim}
## Using project.simplex with spg: fmin= 0.5454545  at
\end{verbatim}
\begin{alltt}
\hlkwd{print}\hlstd{(as}\hlopt{$}\hlstd{par)}
\end{alltt}
\begin{verbatim}
## [1] 0.5454524 0.2727289 0.1818186
\end{verbatim}
\end{kframe}
\end{knitrout}

Apart from the parameter rescaling, this is an entirely "doable" problem. 
Note that we can also solve the problem as a Quadratic Program using
the \pkg{quadprog} package.

\begin{knitrout}\scriptsize
\definecolor{shadecolor}{rgb}{0.969, 0.969, 0.969}\color{fgcolor}\begin{kframe}
\begin{alltt}
\hlkwd{library}\hlstd{(quadprog)}
\hlstd{Dmat}\hlkwb{<-}\hlkwd{diag}\hlstd{(}\hlkwd{c}\hlstd{(}\hlnum{1}\hlstd{,}\hlnum{2}\hlstd{,}\hlnum{3}\hlstd{))}
\hlstd{Amat}\hlkwb{<-}\hlkwd{matrix}\hlstd{(}\hlkwd{c}\hlstd{(}\hlnum{1}\hlstd{,} \hlnum{1}\hlstd{,} \hlnum{1}\hlstd{),} \hlkwc{ncol}\hlstd{=}\hlnum{1}\hlstd{)}
\hlstd{bvec}\hlkwb{<-}\hlkwd{c}\hlstd{(}\hlnum{1}\hlstd{)}
\hlstd{meq}\hlkwb{=}\hlnum{1}
\hlstd{dvec}\hlkwb{<-}\hlkwd{c}\hlstd{(}\hlnum{0}\hlstd{,} \hlnum{0}\hlstd{,} \hlnum{0}\hlstd{)}
\hlstd{ans}\hlkwb{<-}\hlkwd{solve.QP}\hlstd{(Dmat, dvec, Amat, bvec,} \hlkwc{meq}\hlstd{=}\hlnum{0}\hlstd{,} \hlkwc{factorized}\hlstd{=}\hlnum{FALSE}\hlstd{)}
\hlstd{ans}
\end{alltt}
\begin{verbatim}
## $solution
## [1] 0.5454545 0.2727273 0.1818182
## 
## $value
## [1] 0.2727273
## 
## $unconstrained.solution
## [1] 0 0 0
## 
## $iterations
## [1] 2 0
## 
## $Lagrangian
## [1] 0.5454545
## 
## $iact
## [1] 1
\end{verbatim}
\end{kframe}
\end{knitrout}



\section{Conclusion}

Sumscale problems can present difficulties for optimization (or function minimization)
codes. These difficulties are by no means insurmountable, but they do require some 
attention.

While specialized approaches are "best" for speed and correctness, a general user is more
likely to benefit from a simpler approach of embedding the scaling in the objective function
and rescaling the parameters before reporting them. Another choice is to use the projected
gradient via \code{spg} from package \pkg{BB}.



%%\bibliographystyle{chicago} %The style you want to use for references.
\bibliographystyle{abbrvnat}
\bibliography{sumscale} %The files containing all the articles and 


\end{document}

