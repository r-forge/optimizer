\documentclass[10pt]{article}

\usepackage{float}
\usepackage{amsmath} %% Does this have equation*
\usepackage{amssymb}
\usepackage{amsthm}


\usepackage{Sweave}
\begin{document}

%\VignetteIndexEntry{optimr() Extension}
%\VignetteDepends{numDeriv, setRNG}
%\VignetteKeywords{extending optimr, additional optimizer}
%\VignettePackage{optimz}


\section{Extending the optimr() function in package \emph{optimz}}
``optimz'' is a package intended to provide improved and extended 
function minimization tools for \emph{R}. Such facilities are
commonly referred to as ``optimization'', but the original \texttt{optim()}
function and its replacement in this package, that is, \texttt{optimr()},
only allow for the minimization or maximization of nonlinear functions of
multiple parameters subject to at most bounds constraints. Some methods
also permit fixed (masked) parameters.

\begin{Schunk}
\begin{Sinput}
> require("setRNG") 
> setRNG(list(kind="Wichmann-Hill", normal.kind="Box-Muller", seed=1236))
\end{Sinput}
\end{Schunk}


\section{How to optimize a nonlinear objective function with BB?}
The basic function for optimization is \emph{spg}.  It can solve smooth, 
nonlinear optimization problems with box-constraints, and also other types of 
constraints using projection.  
We would like to direct the user to the help page for many examples of 
how to use \emph{spg}.  Here we discuss an example involving estimation of 
parameters maximizing a log-likelihood function for 
a binary Poisson mixture distribution.  


\end{document}
