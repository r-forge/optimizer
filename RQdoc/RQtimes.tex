\documentclass[10pt]{article}\usepackage[]{graphicx}\usepackage[]{color}
%% maxwidth is the original width if it is less than linewidth
%% otherwise use linewidth (to make sure the graphics do not exceed the margin)
\makeatletter
\def\maxwidth{ %
  \ifdim\Gin@nat@width>\linewidth
    \linewidth
  \else
    \Gin@nat@width
  \fi
}
\makeatother

\definecolor{fgcolor}{rgb}{0.345, 0.345, 0.345}
\newcommand{\hlnum}[1]{\textcolor[rgb]{0.686,0.059,0.569}{#1}}%
\newcommand{\hlstr}[1]{\textcolor[rgb]{0.192,0.494,0.8}{#1}}%
\newcommand{\hlcom}[1]{\textcolor[rgb]{0.678,0.584,0.686}{\textit{#1}}}%
\newcommand{\hlopt}[1]{\textcolor[rgb]{0,0,0}{#1}}%
\newcommand{\hlstd}[1]{\textcolor[rgb]{0.345,0.345,0.345}{#1}}%
\newcommand{\hlkwa}[1]{\textcolor[rgb]{0.161,0.373,0.58}{\textbf{#1}}}%
\newcommand{\hlkwb}[1]{\textcolor[rgb]{0.69,0.353,0.396}{#1}}%
\newcommand{\hlkwc}[1]{\textcolor[rgb]{0.333,0.667,0.333}{#1}}%
\newcommand{\hlkwd}[1]{\textcolor[rgb]{0.737,0.353,0.396}{\textbf{#1}}}%
\let\hlipl\hlkwb

\usepackage{framed}
\makeatletter
\newenvironment{kframe}{%
 \def\at@end@of@kframe{}%
 \ifinner\ifhmode%
  \def\at@end@of@kframe{\end{minipage}}%
  \begin{minipage}{\columnwidth}%
 \fi\fi%
 \def\FrameCommand##1{\hskip\@totalleftmargin \hskip-\fboxsep
 \colorbox{shadecolor}{##1}\hskip-\fboxsep
     % There is no \\@totalrightmargin, so:
     \hskip-\linewidth \hskip-\@totalleftmargin \hskip\columnwidth}%
 \MakeFramed {\advance\hsize-\width
   \@totalleftmargin\z@ \linewidth\hsize
   \@setminipage}}%
 {\par\unskip\endMakeFramed%
 \at@end@of@kframe}
\makeatother

\definecolor{shadecolor}{rgb}{.97, .97, .97}
\definecolor{messagecolor}{rgb}{0, 0, 0}
\definecolor{warningcolor}{rgb}{1, 0, 1}
\definecolor{errorcolor}{rgb}{1, 0, 0}
\newenvironment{knitrout}{}{} % an empty environment to be redefined in TeX

\usepackage{alltt}
\usepackage{fullpage}
\usepackage{graphicx} %If you want to include postscript graphics
\usepackage{wrapfig} 
%\usepackage{mystyle} %Create your own file, mystyle.sty where you put all your own \newcommand statements, for example.

\usepackage{chicago}

\raggedbottom
\newcommand{\R}{{\sf R\ }}
%% \newcommand{\V}{\fontsize{\tiny}\\ {\fontsize{\normalsize}}}
%% \newcommand{\eq}{ \begin{equation} }

%% \newcommand{\ee}{ \end{equation} }  These did not work.

%% \newenvironment{knitrout}{}{} % an empty environment to be redefined in TeX
%% suddenly was lacking this in tex file?? 120629

\newcommand{\B}[1]{{\bf #1 \rm}}

\newcommand{\Splus}{{\sf S-PLUS}}
\newcommand{\fixme}[1]{\textbf{FIXME: #1}}
%\newcommand{\fixme}[1]{}
\newcommand{\code}[1]{{\tt #1}}
\newcommand{\pkg}[1]{\bf{\tt #1}\rm }
\IfFileExists{upquote.sty}{\usepackage{upquote}}{}
\begin{document}

\begin{knitrout}
\definecolor{shadecolor}{rgb}{0.969, 0.969, 0.969}\color{fgcolor}\begin{kframe}


{\ttfamily\noindent\color{warningcolor}{\#\# Warning in parse\_objects(paths[1L]): file cache/\_\_objects not found}}

{\ttfamily\noindent\color{warningcolor}{\#\# Warning in parse\_objects(paths[2L]): file cache/\_\_globals not found}}\end{kframe}
\end{knitrout}

\title{Timing Rayleigh Quotient minimization in \R}
\author{John C. Nash, Telfer School of Management, University of Ottawa}
\date{July 2012, with update December 2016}
\maketitle

\section*{Abstract}

This vignette is simply to record the methods and results for timing 
various Rayleigh Quotient minimizations with \R using different functions
and different ways of running the computations, in particular trying Fortran
subroutines and the \R byte compiler.

\section{The computational task}

The maximal and minimal eigensolutions of a symmetric matrix $A$ are extrema of the Rayleigh Quotient

$ R(x) =  (x' A x)  / (x' x) $

We could also deal with generalized eigenproblems of the form 

$A x = e B x$

where B is symmetric and positive definite by using the Rayleigh Quotient (RQ)

$ R_g(x) =  (x' A x)  / (x' B x) $

In this document, B will always be an identity matrix, but some programs we test
assume that it is present.

Noth that the objective is scaled by the parameters, in fact by by their 
sum of squares. Alternatively, 
we may think of requiring the \B{normalized} eigensolution, which is given as 

$ x_{normalized} = x/sqrt(x' x) $

\section{Timings and speedups}

In \R, execution times can be measured by the function \code{system.time},
and in particular the first element of the object this function returns the time
taken by the code which is the argument to the function. 
However, various factors influence computing times in a modern computational
system, so we generally want to run replications of the times. The \R packages
\pkg{rbenchmark} and \pkg{microbenchmark} can be used for this. I have a 
preference for the latter. However, to keep the time to prepare this vignette
with \pkg{Sweave} or \pkg{knitR} reasonable, many of the timings will be
done with only \code{system.time}.

There are some ways to speed up \R computations. 

\begin{itemize}
\item{The code can be modified to use more efficient language structures. We
show some of these below, in particular, to use vector operations.}
\item{We can use the \R byte code compiler by Luke Tierney, which has been 
part of the \R distribution since version 2.14.}
\item{We can use compiled code in other languages. Here we show how Fortran
subroutines can be used.}
\end{itemize}

Note that the timings here are intended to provide a guide to the relative 
efficiency of equivalent computations of the same results. There are hardware,
operating system, library, and package effects that we will largely ignore.
The particular machine used to develop this article is described using 
the following script.

\begin{knitrout}\scriptsize
\definecolor{shadecolor}{rgb}{0.969, 0.969, 0.969}\color{fgcolor}\begin{kframe}
\begin{alltt}
\hlkwd{sessionInfo}\hlstd{()}
\end{alltt}
\begin{verbatim}
## R version 3.4.0 (2017-04-21)
## Platform: x86_64-pc-linux-gnu (64-bit)
## Running under: Linux Mint 18
## 
## Matrix products: default
## BLAS: /usr/lib/libblas/libblas.so.3.6.0
## LAPACK: /usr/lib/lapack/liblapack.so.3.6.0
## 
## locale:
##  [1] LC_CTYPE=en_CA.UTF-8       LC_NUMERIC=C              
##  [3] LC_TIME=en_CA.UTF-8        LC_COLLATE=en_CA.UTF-8    
##  [5] LC_MONETARY=en_CA.UTF-8    LC_MESSAGES=en_CA.UTF-8   
##  [7] LC_PAPER=en_CA.UTF-8       LC_NAME=C                 
##  [9] LC_ADDRESS=C               LC_TELEPHONE=C            
## [11] LC_MEASUREMENT=en_CA.UTF-8 LC_IDENTIFICATION=C       
## 
## attached base packages:
## [1] stats     graphics  grDevices utils     datasets  methods   base     
## 
## other attached packages:
## [1] knitr_1.15.1
## 
## loaded via a namespace (and not attached):
## [1] compiler_3.4.0 magrittr_1.5   tools_3.4.0    stringi_1.1.5 
## [5] digest_0.6.12  stringr_1.2.0  evaluate_0.10
\end{verbatim}
\begin{alltt}
\hlkwd{print}\hlstd{(}\hlkwd{system}\hlstd{(}\hlstr{"inxi"}\hlstd{))} \hlcom{# special bash script for Linux}
\end{alltt}
\begin{verbatim}
## [1] 0
\end{verbatim}
\end{kframe}
\end{knitrout}

The results presented in this article will very likely differ on other 
machinery and other operating environments. However, the code is included -- you
are more than welcome to re-run the timings, and I would be delighted to learn
of any significant variations from the general findings reported here, especially
if the source of such variations can be discovered. 


\section{Our example matrix}

We will use a matrix called the Moler matrix \cite[Appendix 1]{cnm79}. 
This is a positive definite
symmetric matrix with one small eigenvalue. We will show a couple of 
examples of computing the small eigenvalue solution, but will mainly
perform timings using the maximal eigenvalue solution, which we will
find by minimizing the RQ of (-1) times the matrix. (The eigenvalue
of this matrix is the negative of the maximal eigenvalue of the 
original, but the eigenvectors are equivalent to within a scaling
factor for non-degenerate eigenvalues.)

Here is the code for generating the Moler matrix.

\begin{knitrout}\scriptsize
\definecolor{shadecolor}{rgb}{0.969, 0.969, 0.969}\color{fgcolor}\begin{kframe}
\begin{alltt}
\hlstd{molermat}\hlkwb{<-}\hlkwa{function}\hlstd{(}\hlkwc{n}\hlstd{)\{}
   \hlstd{A}\hlkwb{<-}\hlkwd{matrix}\hlstd{(}\hlnum{NA}\hlstd{,} \hlkwc{nrow}\hlstd{=n,} \hlkwc{ncol}\hlstd{=n)}
   \hlkwa{for} \hlstd{(i} \hlkwa{in} \hlnum{1}\hlopt{:}\hlstd{n)\{}
      \hlkwa{for} \hlstd{(j} \hlkwa{in} \hlnum{1}\hlopt{:}\hlstd{n) \{}
          \hlkwa{if} \hlstd{(i} \hlopt{==} \hlstd{j) A[i,i]}\hlkwb{<-}\hlstd{i}
          \hlkwa{else} \hlstd{A[i,j]}\hlkwb{<-}\hlkwd{min}\hlstd{(i,j)} \hlopt{-} \hlnum{2}
      \hlstd{\}}
   \hlstd{\}}
   \hlstd{A}
\hlstd{\}}
\end{alltt}
\end{kframe}
\end{knitrout}

However, since \R is more efficient with vectorized code, the following routine by 
Ravi Varadhan should do much better.

\begin{knitrout}\scriptsize
\definecolor{shadecolor}{rgb}{0.969, 0.969, 0.969}\color{fgcolor}\begin{kframe}
\begin{alltt}
\hlstd{molerfast} \hlkwb{<-} \hlkwa{function}\hlstd{(}\hlkwc{n}\hlstd{) \{}
\hlcom{# A fast version of `molermat'}
\hlstd{A} \hlkwb{<-} \hlkwd{matrix}\hlstd{(}\hlnum{0}\hlstd{,} \hlkwc{nrow} \hlstd{= n,} \hlkwc{ncol} \hlstd{= n)}
\hlstd{j} \hlkwb{<-} \hlnum{1}\hlopt{:}\hlstd{n}
\hlkwa{for} \hlstd{(i} \hlkwa{in} \hlnum{1}\hlopt{:}\hlstd{n) \{}
\hlstd{A[i,} \hlnum{1}\hlopt{:}\hlstd{i]} \hlkwb{<-} \hlkwd{pmin}\hlstd{(i,} \hlnum{1}\hlopt{:}\hlstd{i)} \hlopt{-} \hlnum{2}
\hlstd{\}}
\hlstd{A} \hlkwb{<-} \hlstd{A} \hlopt{+} \hlkwd{t}\hlstd{(A)}
\hlkwd{diag}\hlstd{(A)} \hlkwb{<-} \hlnum{1}\hlopt{:}\hlstd{n}
\hlstd{A}
\hlstd{\}}
\end{alltt}
\end{kframe}
\end{knitrout}

\subsection{Time to build the matrix}

Let us see how long it takes to build the Moler matrix of different sizes.
However, given that it is easy to use the byte-code compiler, we will compare
results. We also include times for the \code{eigen()} function that computes
the full set of eigensolutions very quickly.

\begin{knitrout}\scriptsize
\definecolor{shadecolor}{rgb}{0.969, 0.969, 0.969}\color{fgcolor}\begin{kframe}


{\ttfamily\noindent\itshape\color{messagecolor}{\#\# Loading required package: compiler}}\end{kframe}
\end{knitrout}

\begin{knitrout}\scriptsize
\definecolor{shadecolor}{rgb}{0.969, 0.969, 0.969}\color{fgcolor}\begin{kframe}
\begin{verbatim}
##       n buildi buildc   osize eigentime bfast bfastc
## 1   100  0.028  0.012   80200     0.012 0.020  0.004
## 2   200  0.044  0.040  320200     0.044 0.004  0.008
## 3   300  0.096  0.096  720200     0.132 0.012  0.012
## 4   400  0.168  0.160 1280200     0.288 0.016  0.020
## 5   500  0.256  0.260 2000200     0.536 0.028  0.024
## 6   600  0.368  0.368 2880200     0.984 0.032  0.032
## 7   700  0.516  0.496 3920200     1.500 0.040  0.044
## 8   800  0.660  0.660 5120200     2.196 0.048  0.048
## 9   900  0.836  0.832 6480200     3.140 0.060  0.068
## 10 1000  1.032  1.032 8000200     4.188 0.072  0.076
## buildi - interpreted build time; buildc - byte compiled build time
## osize - matrix size in bytes; eigentime - all eigensolutions time
## bfast - interpreted vectorized build time; bfastc - same code, byte compiled time
\end{verbatim}
\end{kframe}
\end{knitrout}

We can graph the times, and show a definite advantage for
using the byte code compiler. The code, which is not 
echoed here, also models the times and the object size
created as almost perfect quadratic models in \code{n}. However,
the vectorized code is much, much faster, and the byte code
compiler does not appear to help.

\begin{knitrout}\scriptsize
\definecolor{shadecolor}{rgb}{0.969, 0.969, 0.969}\color{fgcolor}
\includegraphics[width=\maxwidth]{figure/drawtime1-1} 

\end{knitrout}

We can also model these timings. If we try to fit a quadratic model in the
matrix size, we find almost perfect fits for both interpreted and byte-compiled
timings over the cases tried.

\begin{knitrout}\scriptsize
\definecolor{shadecolor}{rgb}{0.969, 0.969, 0.969}\color{fgcolor}\begin{kframe}
\begin{alltt}
\hlstd{n2}\hlkwb{<-}\hlstd{n}\hlopt{*}\hlstd{n}
\hlstd{itime}\hlkwb{<-}\hlkwd{lm}\hlstd{(ti}\hlopt{~}\hlstd{n}\hlopt{+}\hlstd{n2)}
\hlkwd{summary}\hlstd{(itime)}
\end{alltt}
\begin{verbatim}
## 
## Call:
## lm(formula = ti ~ n + n2)
## 
## Residuals:
##        Min         1Q     Median         3Q        Max 
## -0.0054545 -0.0027455 -0.0012970  0.0001636  0.0109091 
## 
## Coefficients:
##               Estimate Std. Error t value Pr(>|t|)    
## (Intercept)  1.653e-02  6.446e-03   2.565   0.0373 *  
## n           -4.873e-05  2.692e-05  -1.810   0.1132    
## n2           1.067e-06  2.385e-08  44.725  7.3e-10 ***
## ---
## Signif. codes:  0 '***' 0.001 '**' 0.01 '*' 0.05 '.' 0.1 ' ' 1
## 
## Residual standard error: 0.00548 on 7 degrees of freedom
## Multiple R-squared:  0.9998,	Adjusted R-squared:  0.9998 
## F-statistic: 1.837e+04 on 2 and 7 DF,  p-value: 9.538e-14
\end{verbatim}
\begin{alltt}
\hlstd{ctime}\hlkwb{<-}\hlkwd{lm}\hlstd{(tc}\hlopt{~}\hlstd{n}\hlopt{+}\hlstd{n2)}
\hlkwd{summary}\hlstd{(ctime)}
\end{alltt}
\begin{verbatim}
## 
## Call:
## lm(formula = tc ~ n + n2)
## 
## Residuals:
##        Min         1Q     Median         3Q        Max 
## -0.0062364 -0.0016909 -0.0004182  0.0023773  0.0049455 
## 
## Coefficients:
##               Estimate Std. Error t value Pr(>|t|)    
## (Intercept)  4.600e-03  4.538e-03   1.014    0.345    
## n           -2.409e-05  1.895e-05  -1.271    0.244    
## n2           1.050e-06  1.679e-08  62.529 7.03e-11 ***
## ---
## Signif. codes:  0 '***' 0.001 '**' 0.01 '*' 0.05 '.' 0.1 ' ' 1
## 
## Residual standard error: 0.003859 on 7 degrees of freedom
## Multiple R-squared:  0.9999,	Adjusted R-squared:  0.9999 
## F-statistic: 3.739e+04 on 2 and 7 DF,  p-value: 7.933e-15
\end{verbatim}
\begin{alltt}
\hlstd{osize}\hlkwb{<-}\hlkwd{lm}\hlstd{(os}\hlopt{~}\hlstd{n}\hlopt{+}\hlstd{n2)}
\hlkwd{summary}\hlstd{(osize)}
\end{alltt}


{\ttfamily\noindent\color{warningcolor}{\#\# Warning in summary.lm(osize): essentially perfect fit: summary may be unreliable}}\begin{verbatim}
## 
## Call:
## lm(formula = os ~ n + n2)
## 
## Residuals:
##        Min         1Q     Median         3Q        Max 
## -7.707e-10 -1.136e-11  8.319e-11  1.607e-10  1.963e-10 
## 
## Coefficients:
##               Estimate Std. Error    t value Pr(>|t|)    
## (Intercept)  2.000e+02  3.821e-10  5.235e+11   <2e-16 ***
## n           -4.101e-12  1.596e-12 -2.570e+00    0.037 *  
## n2           8.000e+00  1.414e-15  5.659e+15   <2e-16 ***
## ---
## Signif. codes:  0 '***' 0.001 '**' 0.01 '*' 0.05 '.' 0.1 ' ' 1
## 
## Residual standard error: 3.248e-10 on 7 degrees of freedom
## Multiple R-squared:      1,	Adjusted R-squared:      1 
## F-statistic: 3.188e+32 on 2 and 7 DF,  p-value: < 2.2e-16
\end{verbatim}
\end{kframe}
\end{knitrout}


\section{Computing the Rayleigh Quotient}

\label{sect:crq}

The Rayleigh Quotient requires the quadratic form $ x' A x$ divided
by the inner product $x' x$. \R lets us form this in several ways. 
Given that we know \code{for} loops are slow, we will not actually
use the direct code (incorporating the minus sign)

\begin{knitrout}\scriptsize
\definecolor{shadecolor}{rgb}{0.969, 0.969, 0.969}\color{fgcolor}\begin{kframe}
\begin{alltt}
\hlstd{rqdir}\hlkwb{<-}\hlkwa{function}\hlstd{(}\hlkwc{x}\hlstd{,} \hlkwc{AA}\hlstd{)\{}
  \hlstd{rq}\hlkwb{<-}\hlnum{0.0}
  \hlstd{n}\hlkwb{<-}\hlkwd{length}\hlstd{(x)} \hlcom{# assume x, AA conformable}
  \hlkwa{for} \hlstd{(i} \hlkwa{in} \hlnum{1}\hlopt{:}\hlstd{n) \{}
     \hlkwa{for} \hlstd{(j} \hlkwa{in} \hlnum{1}\hlopt{:}\hlstd{n) \{}
        \hlstd{rq}\hlkwb{<-}\hlstd{rq}\hlopt{-}\hlstd{x[i]}\hlopt{*}\hlstd{AA[[i,j]]}\hlopt{*}\hlstd{x[j]} \hlcom{# Note - sign}
     \hlstd{\}}
  \hlstd{\}}
  \hlstd{rq}
\hlstd{\}}
\end{alltt}
\end{kframe}
\end{knitrout}

Somewhat better (as we shall show below) is

\begin{knitrout}\scriptsize
\definecolor{shadecolor}{rgb}{0.969, 0.969, 0.969}\color{fgcolor}\begin{kframe}
\begin{alltt}
\hlstd{ray1}\hlkwb{<-}\hlkwa{function}\hlstd{(}\hlkwc{x}\hlstd{,} \hlkwc{AA}\hlstd{)\{}
    \hlstd{rq}\hlkwb{<-} \hlopt{-} \hlkwd{t}\hlstd{(x)}\hlopt\hlstd{AA}\hlopt\hlstd{x}
\hlstd{\}}
\end{alltt}
\end{kframe}
\end{knitrout}

and better still is 

\begin{knitrout}\scriptsize
\definecolor{shadecolor}{rgb}{0.969, 0.969, 0.969}\color{fgcolor}\begin{kframe}
\begin{alltt}
\hlstd{ray2}\hlkwb{<-}\hlkwa{function}\hlstd{(}\hlkwc{x}\hlstd{,} \hlkwc{AA}\hlstd{)\{}
    \hlstd{rq}\hlkwb{<-} \hlopt{-} \hlkwd{as.numeric}\hlstd{(}\hlkwd{crossprod}\hlstd{(x,} \hlkwd{crossprod}\hlstd{(AA,x)))}
\hlstd{\}}
\end{alltt}
\end{kframe}
\end{knitrout}

Note that we include the minus sign already in these
routines.

If we already have the inner product $ A x$ as \code{ax} from some other 
computation, then we can simply use 

\begin{knitrout}\scriptsize
\definecolor{shadecolor}{rgb}{0.969, 0.969, 0.969}\color{fgcolor}\begin{kframe}
\begin{alltt}
\hlstd{ray3}\hlkwb{<-}\hlkwa{function}\hlstd{(}\hlkwc{x}\hlstd{,} \hlkwc{AA}\hlstd{,} \hlkwc{ax}\hlstd{=axftn)\{}
    \hlcom{# ax is a function to form AA%*%x }
    \hlstd{rq}\hlkwb{<-} \hlopt{-} \hlkwd{as.numeric}\hlstd{(}\hlkwd{crossprod}\hlstd{(x,} \hlkwd{ax}\hlstd{(x, AA)))}
\hlstd{\}}
\end{alltt}
\end{kframe}
\end{knitrout}


\section{Matrix-vector products}

In generating the RQ, we do not actually need the matrix itself, 
but simply the inner product with a vector \code{x}, from which 
a second inner produce with \code{x} gives us the quadratic form
$ x' A x$. If \code{n} is the order of the problem, then for large
\code{n}, we avoid storing and manipulating a very large matrix if
we use \B{implicit inner product} formation. We do this with the
following code. For future reference, we include the multiplication
by an identity. 

\begin{knitrout}\scriptsize
\definecolor{shadecolor}{rgb}{0.969, 0.969, 0.969}\color{fgcolor}\begin{kframe}
\begin{alltt}
\hlstd{ax}\hlkwb{<-}\hlkwa{function}\hlstd{(}\hlkwc{x}\hlstd{,} \hlkwc{AA}\hlstd{)\{}
   \hlstd{u}\hlkwb{<-} \hlkwd{as.numeric}\hlstd{(AA}\hlopt\hlstd{x)}
\hlstd{\}}

\hlstd{axx}\hlkwb{<-}\hlkwa{function}\hlstd{(}\hlkwc{x}\hlstd{,} \hlkwc{AA}\hlstd{)\{}
   \hlstd{u}\hlkwb{<-} \hlkwd{as.numeric}\hlstd{(}\hlkwd{crossprod}\hlstd{(AA, x))}
\hlstd{\}}
\end{alltt}
\end{kframe}
\end{knitrout}

Note that second argument, supposedly communicating the matrix which is
to be used in the matrix-vector product, is ignored in the following
implicit product routine. It is present only to provide a common syntax
when we wish to try different routines within other computations.

\begin{knitrout}\scriptsize
\definecolor{shadecolor}{rgb}{0.969, 0.969, 0.969}\color{fgcolor}\begin{kframe}
\begin{alltt}
\hlstd{aximp}\hlkwb{<-}\hlkwa{function}\hlstd{(}\hlkwc{x}\hlstd{,} \hlkwc{AA}\hlstd{=}\hlnum{1}\hlstd{)\{} \hlcom{# implicit moler A*x}
   \hlstd{n}\hlkwb{<-}\hlkwd{length}\hlstd{(x)}
   \hlstd{y}\hlkwb{<-}\hlkwd{rep}\hlstd{(}\hlnum{0}\hlstd{,n)}
   \hlkwa{for} \hlstd{(i} \hlkwa{in} \hlnum{1}\hlopt{:}\hlstd{n)\{}
      \hlstd{tt}\hlkwb{<-}\hlnum{0.}
      \hlkwa{for} \hlstd{(j} \hlkwa{in} \hlnum{1}\hlopt{:}\hlstd{n) \{}
          \hlkwa{if} \hlstd{(i} \hlopt{==} \hlstd{j) tt}\hlkwb{<-}\hlstd{tt}\hlopt{+}\hlstd{i}\hlopt{*}\hlstd{x[i]}
          \hlkwa{else} \hlstd{tt}\hlkwb{<-}\hlstd{tt}\hlopt{+}\hlstd{(}\hlkwd{min}\hlstd{(i,j)} \hlopt{-} \hlnum{2}\hlstd{)}\hlopt{*}\hlstd{x[j]}
      \hlstd{\}}
      \hlstd{y[i]}\hlkwb{<-}\hlstd{tt}
   \hlstd{\}}
   \hlstd{y}
\hlstd{\}}
\hlstd{ident}\hlkwb{<-}\hlkwa{function}\hlstd{(}\hlkwc{x}\hlstd{,} \hlkwc{B}\hlstd{=}\hlnum{1}\hlstd{) x} \hlcom{# identity}
\end{alltt}
\end{kframe}
\end{knitrout}

However, Ravi Varadhan has suggested the following vectorized code for
the implicit matrix-vector product.

\begin{knitrout}\scriptsize
\definecolor{shadecolor}{rgb}{0.969, 0.969, 0.969}\color{fgcolor}\begin{kframe}
\begin{alltt}
\hlstd{axmolerfast} \hlkwb{<-} \hlkwa{function}\hlstd{(}\hlkwc{x}\hlstd{,} \hlkwc{AA}\hlstd{=}\hlnum{1}\hlstd{) \{}
\hlcom{# A fast and memory-saving version of A%*%x  }
\hlcom{# For Moler matrix. Note we need a matrix argument to match other functions}
\hlstd{n} \hlkwb{<-} \hlkwd{length}\hlstd{(x)}
\hlstd{j} \hlkwb{<-} \hlnum{1}\hlopt{:}\hlstd{n}
\hlstd{ax} \hlkwb{<-} \hlkwd{rep}\hlstd{(}\hlnum{0}\hlstd{, n)}
\hlkwa{for} \hlstd{(i} \hlkwa{in} \hlnum{1}\hlopt{:}\hlstd{n) \{}
\hlstd{term} \hlkwb{<-} \hlstd{x} \hlopt{*} \hlstd{(}\hlkwd{pmin}\hlstd{(i, j)} \hlopt{-} \hlnum{2}\hlstd{)}
\hlstd{ax[i]} \hlkwb{<-} \hlkwd{sum}\hlstd{(term[}\hlopt{-}\hlstd{i])}
\hlstd{\}}
\hlstd{ax} \hlkwb{<-} \hlstd{ax} \hlopt{+} \hlstd{j}\hlopt{*}\hlstd{x}
\hlstd{ax}
\hlstd{\}}
\end{alltt}
\end{kframe}
\end{knitrout}

We can also use external language routines, for example in Fortran. 
However, this needs a Fortran \B{subroutine} which outputs the result as one of
the returned components. The subroutine is in file \code{moler.f}.

\begin{verbatim}
      subroutine moler(n, x, ax)
      integer n, i, j
      double precision x(n), ax(n), sum
c     return ax = A * x for A = moler matrix
c     A[i,j]=min(i,j)-2 for i<>j, or i for i==j
      do 20 i=1,n
         sum=0.0
         do 10 j=1,n
            if (i.eq.j) then
               sum = sum+i*x(i)
            else
               sum = sum+(min(i,j)-2)*x(j)
            endif
 10      continue
         ax(i)=sum
 20   continue
      return
      end
\end{verbatim}

This is then compiled in a form suitable for R use by the command:

\begin{verbatim}
R CMD SHLIB moler.f
\end{verbatim}

This creates files \code{moler.o} and \code{moler.so}, the latter being the
dynamically loadable library we need to bring into our \R session. Normally
the compilation is run as a command-line tool, and at first was run in Ubuntu Linux 
in a directory containing
the file \code{moler.f} but outside this vignette. When I came to check and possibly
update this file, I discovered that there were some minor syntax changes in the
\textbf{knitr} package so that directives for \code{cache} caused errors and needed
to be deleted. Also the dynamic load library files \code{moler.o} and \code{moler.so}
would not load, apparently because they were compiled using different versions of
the Fortran libraries. Recompiling removed this issue. However, we can automate the
process to avoid future difficulties as the operating system infrastructure is 
updated. 

\begin{knitrout}\scriptsize
\definecolor{shadecolor}{rgb}{0.969, 0.969, 0.969}\color{fgcolor}\begin{kframe}
\begin{alltt}
\hlkwd{system}\hlstd{(}\hlstr{"rm moler.so"}\hlstd{)}
\hlkwd{system}\hlstd{(}\hlstr{"rm moler.o"}\hlstd{)}
\hlkwd{system}\hlstd{(}\hlstr{"R CMD SHLIB moler.f"}\hlstd{)}
\hlkwd{cat}\hlstd{(}\hlstr{"Dynamic libraries rebuilt \textbackslash{}n"}\hlstd{)}
\end{alltt}
\begin{verbatim}
## Dynamic libraries rebuilt
\end{verbatim}
\end{kframe}
\end{knitrout}


\begin{knitrout}\scriptsize
\definecolor{shadecolor}{rgb}{0.969, 0.969, 0.969}\color{fgcolor}\begin{kframe}
\begin{alltt}
\hlkwd{dyn.load}\hlstd{(}\hlstr{"moler.so"}\hlstd{)}
\hlkwd{cat}\hlstd{(}\hlstr{"Is the mat multiply loaded? "}\hlstd{,}\hlkwd{is.loaded}\hlstd{(}\hlstr{"moler"}\hlstd{),}\hlstr{"\textbackslash{}n"}\hlstd{)}
\end{alltt}
\begin{verbatim}
## Is the mat multiply loaded?  TRUE
\end{verbatim}
\begin{alltt}
\hlstd{axftn}\hlkwb{<-}\hlkwa{function}\hlstd{(}\hlkwc{x}\hlstd{,} \hlkwc{AA}\hlstd{=}\hlnum{1}\hlstd{) \{} \hlcom{# ignore second argument}
   \hlstd{n}\hlkwb{<-}\hlkwd{length}\hlstd{(x)} \hlcom{# could speed up by having this passed}
   \hlstd{vout}\hlkwb{<-}\hlkwd{rep}\hlstd{(}\hlnum{0}\hlstd{,n)} \hlcom{# purely for storage}
   \hlstd{res}\hlkwb{<-}\hlstd{(}\hlkwd{.Fortran}\hlstd{(}\hlstr{"moler"}\hlstd{,} \hlkwc{n}\hlstd{=}\hlkwd{as.integer}\hlstd{(n),} \hlkwc{x}\hlstd{=}\hlkwd{as.double}\hlstd{(x),} \hlkwc{vout}\hlstd{=}\hlkwd{as.double}\hlstd{(vout)))}\hlopt{$}\hlstd{vout}
\hlstd{\}}
\end{alltt}
\end{kframe}
\end{knitrout}


We can also byte compile each of the routines above

\begin{knitrout}\scriptsize
\definecolor{shadecolor}{rgb}{0.969, 0.969, 0.969}\color{fgcolor}\begin{kframe}
\begin{alltt}
\hlkwd{require}\hlstd{(compiler)}
\hlstd{axc}\hlkwb{<-}\hlkwd{cmpfun}\hlstd{(ax)}
\hlstd{axxc}\hlkwb{<-}\hlkwd{cmpfun}\hlstd{(axx)}
\hlstd{axftnc}\hlkwb{<-}\hlkwd{cmpfun}\hlstd{(axftn)}
\hlstd{aximpc}\hlkwb{<-}\hlkwd{cmpfun}\hlstd{(aximp)}
\hlstd{axmfc}\hlkwb{<-}\hlkwd{cmpfun}\hlstd{(axmolerfast)}
\end{alltt}
\end{kframe}
\end{knitrout}

Now it is possible to time the different approaches to the matrix-vector
product. We only use matrix sizes up to 500 here. 

\begin{knitrout}\scriptsize
\definecolor{shadecolor}{rgb}{0.969, 0.969, 0.969}\color{fgcolor}\begin{kframe}
\begin{alltt}
\hlkwd{dyn.load}\hlstd{(}\hlstr{"moler.so"}\hlstd{)}
\hlkwd{cat}\hlstd{(}\hlstr{"Is the mat multiply loaded? "}\hlstd{,}\hlkwd{is.loaded}\hlstd{(}\hlstr{"moler"}\hlstd{),}\hlstr{"\textbackslash{}n"}\hlstd{)}
\end{alltt}
\begin{verbatim}
## Is the mat multiply loaded?  TRUE
\end{verbatim}
\begin{alltt}
\hlcom{# require(microbenchmark)}
\hlstd{nmax}\hlkwb{<-}\hlnum{10}
\hlstd{ptable}\hlkwb{<-}\hlkwd{matrix}\hlstd{(}\hlnum{NA}\hlstd{,} \hlkwc{nrow}\hlstd{=nmax,} \hlkwc{ncol}\hlstd{=}\hlnum{11}\hlstd{)} \hlcom{# to hold results}
\hlcom{# loop over sizes}
\hlkwa{for} \hlstd{(ni} \hlkwa{in} \hlnum{1}\hlopt{:}\hlstd{nmax)\{}
  \hlstd{n}\hlkwb{<-}\hlnum{100}\hlopt{*}\hlstd{ni}
  \hlstd{x}\hlkwb{<-}\hlkwd{runif}\hlstd{(n)} \hlcom{# generate a vector }
  \hlstd{ptable[[ni,} \hlnum{1}\hlstd{]]}\hlkwb{<-}\hlstd{n}
  \hlstd{AA}\hlkwb{<-}\hlkwd{molermat}\hlstd{(n)}
  \hlstd{tax}\hlkwb{<-}\hlkwd{system.time}\hlstd{(oax}\hlkwb{<-}\hlkwd{replicate}\hlstd{(}\hlnum{20}\hlstd{,}\hlkwd{ax}\hlstd{(x, AA))[,}\hlnum{1}\hlstd{])[[}\hlnum{1}\hlstd{]]}
  \hlstd{taxc}\hlkwb{<-}\hlkwd{system.time}\hlstd{(oaxc}\hlkwb{<-}\hlkwd{replicate}\hlstd{(}\hlnum{20}\hlstd{,}\hlkwd{axc}\hlstd{(x, AA))[,}\hlnum{1}\hlstd{])[[}\hlnum{1}\hlstd{]]}
  \hlkwa{if} \hlstd{(}\hlopt{!} \hlkwd{identical}\hlstd{(oax, oaxc))} \hlkwd{stop}\hlstd{(}\hlstr{"oaxc NOT correct"}\hlstd{)}
  \hlstd{taxx}\hlkwb{<-}\hlkwd{system.time}\hlstd{(oaxx}\hlkwb{<-}\hlkwd{replicate}\hlstd{(}\hlnum{20}\hlstd{,}\hlkwd{axx}\hlstd{(x, AA))[,}\hlnum{1}\hlstd{])[[}\hlnum{1}\hlstd{]]}
  \hlkwa{if} \hlstd{(}\hlopt{!} \hlkwd{identical}\hlstd{(oax, oaxx))} \hlkwd{stop}\hlstd{(}\hlstr{"oaxx NOT correct"}\hlstd{)}
  \hlstd{taxxc}\hlkwb{<-}\hlkwd{system.time}\hlstd{(oaxxc}\hlkwb{<-}\hlkwd{replicate}\hlstd{(}\hlnum{20}\hlstd{,}\hlkwd{axxc}\hlstd{(x, AA))[,}\hlnum{1}\hlstd{])[[}\hlnum{1}\hlstd{]]}
  \hlkwa{if} \hlstd{(}\hlopt{!} \hlkwd{identical}\hlstd{(oax, oaxxc))} \hlkwd{stop}\hlstd{(}\hlstr{"oaxxc NOT correct"}\hlstd{)}
  \hlstd{taxftn}\hlkwb{<-}\hlkwd{system.time}\hlstd{(oaxftn}\hlkwb{<-}\hlkwd{replicate}\hlstd{(}\hlnum{20}\hlstd{,}\hlkwd{axftn}\hlstd{(x,} \hlkwc{AA}\hlstd{=}\hlnum{1}\hlstd{))[,}\hlnum{1}\hlstd{])[[}\hlnum{1}\hlstd{]]}
  \hlkwa{if} \hlstd{(}\hlopt{!} \hlkwd{identical}\hlstd{(oax, oaxftn))} \hlkwd{stop}\hlstd{(}\hlstr{"oaxftn NOT correct"}\hlstd{)}
  \hlstd{taxftnc}\hlkwb{<-}\hlkwd{system.time}\hlstd{(oaxftnc}\hlkwb{<-}\hlkwd{replicate}\hlstd{(}\hlnum{20}\hlstd{,}\hlkwd{axftnc}\hlstd{(x,} \hlkwc{AA}\hlstd{=}\hlnum{1}\hlstd{))[,}\hlnum{1}\hlstd{])[[}\hlnum{1}\hlstd{]]}
  \hlkwa{if} \hlstd{(}\hlopt{!} \hlkwd{identical}\hlstd{(oax, oaxftnc))} \hlkwd{stop}\hlstd{(}\hlstr{"oaxftnc NOT correct"}\hlstd{)}
  \hlstd{taximp}\hlkwb{<-}\hlkwd{system.time}\hlstd{(oaximp}\hlkwb{<-}\hlkwd{replicate}\hlstd{(}\hlnum{20}\hlstd{,}\hlkwd{aximp}\hlstd{(x,} \hlkwc{AA}\hlstd{=}\hlnum{1}\hlstd{))[,}\hlnum{1}\hlstd{])[[}\hlnum{1}\hlstd{]]}
  \hlkwa{if} \hlstd{(}\hlopt{!} \hlkwd{identical}\hlstd{(oax, oaximp))} \hlkwd{stop}\hlstd{(}\hlstr{"oaximp NOT correct"}\hlstd{)}
  \hlstd{taximpc}\hlkwb{<-}\hlkwd{system.time}\hlstd{(oaximpc}\hlkwb{<-}\hlkwd{replicate}\hlstd{(}\hlnum{20}\hlstd{,}\hlkwd{aximpc}\hlstd{(x,} \hlkwc{AA}\hlstd{=}\hlnum{1}\hlstd{))[,}\hlnum{1}\hlstd{])[[}\hlnum{1}\hlstd{]]}
  \hlkwa{if} \hlstd{(}\hlopt{!} \hlkwd{identical}\hlstd{(oax, oaximpc))} \hlkwd{stop}\hlstd{(}\hlstr{"oaximpc NOT correct"}\hlstd{)}
  \hlstd{taxmfi}\hlkwb{<-}\hlkwd{system.time}\hlstd{(oaxmfi}\hlkwb{<-}\hlkwd{replicate}\hlstd{(}\hlnum{20}\hlstd{,}\hlkwd{axmolerfast}\hlstd{(x,} \hlkwc{AA}\hlstd{=}\hlnum{1}\hlstd{))[,}\hlnum{1}\hlstd{])[[}\hlnum{1}\hlstd{]]}
  \hlkwa{if} \hlstd{(}\hlopt{!} \hlkwd{identical}\hlstd{(oax, oaxmfi))} \hlkwd{stop}\hlstd{(}\hlstr{"oaxmfi NOT correct"}\hlstd{)}
  \hlstd{taxmfc}\hlkwb{<-}\hlkwd{system.time}\hlstd{(oaxmfc}\hlkwb{<-}\hlkwd{replicate}\hlstd{(}\hlnum{20}\hlstd{,}\hlkwd{axmfc}\hlstd{(x,} \hlkwc{AA}\hlstd{=}\hlnum{1}\hlstd{))[,}\hlnum{1}\hlstd{])[[}\hlnum{1}\hlstd{]]}
  \hlkwa{if} \hlstd{(}\hlopt{!} \hlkwd{identical}\hlstd{(oax, oaxmfc))} \hlkwd{stop}\hlstd{(}\hlstr{"oaxmfc NOT correct"}\hlstd{)}
  \hlstd{ptable[[ni,} \hlnum{2}\hlstd{]]}\hlkwb{<-}\hlstd{tax}
  \hlstd{ptable[[ni,} \hlnum{3}\hlstd{]]}\hlkwb{<-}\hlstd{taxc}
  \hlstd{ptable[[ni,} \hlnum{4}\hlstd{]]}\hlkwb{<-}\hlstd{taxx}
  \hlstd{ptable[[ni,} \hlnum{5}\hlstd{]]}\hlkwb{<-}\hlstd{taxxc}
  \hlstd{ptable[[ni,} \hlnum{6}\hlstd{]]}\hlkwb{<-}\hlstd{taxftn}
  \hlstd{ptable[[ni,} \hlnum{7}\hlstd{]]}\hlkwb{<-}\hlstd{taxftnc}
  \hlstd{ptable[[ni,} \hlnum{8}\hlstd{]]}\hlkwb{<-}\hlstd{taximp}
  \hlstd{ptable[[ni,} \hlnum{9}\hlstd{]]}\hlkwb{<-}\hlstd{taximpc}
  \hlstd{ptable[[ni,} \hlnum{10}\hlstd{]]}\hlkwb{<-}\hlstd{taxmfi}
  \hlstd{ptable[[ni,} \hlnum{11}\hlstd{]]}\hlkwb{<-}\hlstd{taxmfc}
\hlcom{#  cat(n,tax, taxc, taxx, taxxc, taxftn, taxftnc, taximp, taximpc,"\textbackslash{}n")}
\hlstd{\}}
\hlstd{axtym}\hlkwb{<-}\hlkwd{data.frame}\hlstd{(}\hlkwc{n}\hlstd{=ptable[,}\hlnum{1}\hlstd{],} \hlkwc{ax}\hlstd{=ptable[,}\hlnum{2}\hlstd{],} \hlkwc{axc}\hlstd{=ptable[,}\hlnum{3}\hlstd{],}
  \hlkwc{axx}\hlstd{=ptable[,}\hlnum{4}\hlstd{],} \hlkwc{axxc}\hlstd{=ptable[,}\hlnum{5}\hlstd{],}
  \hlkwc{axftn}\hlstd{=ptable[,}\hlnum{6}\hlstd{],} \hlkwc{axftnc}\hlstd{=ptable[,}\hlnum{7}\hlstd{],}
  \hlkwc{aximp}\hlstd{=ptable[,}\hlnum{8}\hlstd{],} \hlkwc{aximpc}\hlstd{=ptable[,}\hlnum{9}\hlstd{],} \hlkwc{axmfast}\hlstd{=ptable[,}\hlnum{10}\hlstd{],}
  \hlkwc{amfastc}\hlstd{=ptable[,}\hlnum{11}\hlstd{])}
\hlkwd{print}\hlstd{(axtym)}
\end{alltt}
\begin{verbatim}
##       n    ax   axc   axx  axxc axftn axftnc  aximp aximpc axmfast amfastc
## 1   100 0.008 0.004 0.004 0.004 0.012  0.008  0.240  0.204   0.084   0.064
## 2   200 0.000 0.004 0.000 0.004 0.004  0.004  0.908  0.828   0.132   0.144
## 3   300 0.004 0.008 0.008 0.008 0.012  0.008  1.856  1.856   0.252   0.240
## 4   400 0.012 0.012 0.012 0.012 0.016  0.016  3.276  3.340   0.360   0.360
## 5   500 0.016 0.016 0.020 0.020 0.028  0.024  5.136  5.196   0.488   0.484
## 6   600 0.028 0.028 0.028 0.028 0.036  0.036  7.424  7.448   0.632   0.628
## 7   700 0.044 0.044 0.044 0.044 0.048  0.048 10.072 10.068   0.780   0.784
## 8   800 0.056 0.056 0.060 0.056 0.064  0.060 13.160 12.900   1.028   0.956
## 9   900 0.072 0.076 0.076 0.076 0.076  0.080 16.228 16.228   1.140   1.140
## 10 1000 0.088 0.092 0.092 0.092 0.096  0.096 19.968 20.004   1.340   1.344
\end{verbatim}
\end{kframe}
\end{knitrout}

From the above output, we see that the \code{crossprod} variant of the 
matrix-vector product appears to be the fastest. However, we have omitted
the time to build the matrix. If we must build the matrix, then we need 
somehow to include that time. Because the times for the matrix-vector 
product were so short, we used \code{replicate} above to run 20 copies 
of the same calculation, which may give some distortion of the timings.
However, we believe the scale of the times is more or less correct. To 
compare these times to the times for the Fortran or implicit matrix-vector
routines, we should add a multiple of the relevant interpreted or
compiled build times. Here we have used the times for the rather poor
\code{molermat()} function, but this is simply to illustrate the range
of potential timings. Apportioning such "fixed costs" to timings is
never a trivial decision. Similarly if, where and how to store
large matrices if we do build them, and whether it is worth building
them more than once if storage is an issue, are all questions that 
may need to be addressed if performance becomes important.

\begin{knitrout}\scriptsize
\definecolor{shadecolor}{rgb}{0.969, 0.969, 0.969}\color{fgcolor}\begin{kframe}
\begin{alltt}
\hlstd{bmattym} \hlkwb{<-} \hlstd{bmattym[}\hlnum{1}\hlopt{:}\hlnum{10}\hlstd{,]}
\hlstd{adjtym}\hlkwb{<-}\hlkwd{data.frame}\hlstd{(}\hlkwc{n}\hlstd{=axtym}\hlopt{$}\hlstd{n,} \hlkwc{axx1}\hlstd{=axtym}\hlopt{$}\hlstd{axx}\hlopt{+}\hlnum{1}\hlopt{*}\hlstd{bmattym}\hlopt{$}\hlstd{buildi,}
     \hlkwc{axxz}\hlstd{=axtym}\hlopt{$}\hlstd{axx}\hlopt{+}\hlnum{20}\hlopt{*}\hlstd{bmattym}\hlopt{$}\hlstd{buildi,}
     \hlkwc{axxc1}\hlstd{=axtym}\hlopt{$}\hlstd{axxc}\hlopt{+}\hlnum{1}\hlopt{*}\hlstd{bmattym}\hlopt{$}\hlstd{buildc,}\hlkwc{axxcz}\hlstd{=axtym}\hlopt{$}\hlstd{axxc}\hlopt{+}\hlnum{20}\hlopt{*}\hlstd{bmattym}\hlopt{$}\hlstd{buildc,}
     \hlkwc{axftn}\hlstd{=axtym}\hlopt{$}\hlstd{axftn,} \hlkwc{aximp}\hlstd{=axtym}\hlopt{$}\hlstd{aximp,} \hlkwc{aximpc}\hlstd{=axtym}\hlopt{$}\hlstd{aximpc)}
\hlkwd{print}\hlstd{(adjtym)}
\end{alltt}
\begin{verbatim}
##       n  axx1   axxz axxc1  axxcz axftn  aximp aximpc
## 1   100 0.032  0.564 0.016  0.244 0.012  0.240  0.204
## 2   200 0.044  0.880 0.044  0.804 0.004  0.908  0.828
## 3   300 0.104  1.928 0.104  1.928 0.012  1.856  1.856
## 4   400 0.180  3.372 0.172  3.212 0.016  3.276  3.340
## 5   500 0.276  5.140 0.280  5.220 0.028  5.136  5.196
## 6   600 0.396  7.388 0.396  7.388 0.036  7.424  7.448
## 7   700 0.560 10.364 0.540  9.964 0.048 10.072 10.068
## 8   800 0.720 13.260 0.716 13.256 0.064 13.160 12.900
## 9   900 0.912 16.796 0.908 16.716 0.076 16.228 16.228
## 10 1000 1.124 20.732 1.124 20.732 0.096 19.968 20.004
\end{verbatim}
\end{kframe}
\end{knitrout}

Out of all this, we see that the Fortran implicit matrix-vector product is 
the overall winner at all values of \code{n}. Moreover, it does NOT 
require the creation and storage of the matrix. However, using Fortran
does involve rather more work for the user, and for most applications 
it is likely we could live with the use of either 

\begin{itemize}
\item{the interpreted matrix-product based on \code{crossprod} 
and an actual matrix is good enough, especially if a fast 
matrix build is used and we have plenty of memory, or}
\item{the interpreted or byte-code compiled implicit matrix-vector 
multiply \code{axmolerfast}.} 
\end{itemize}


\section{RQ computation times}

We have in Section \ref{sect:crq} above set up three versions of a Rayleigh 
Quotient calculation in addition to the direct form. The third 
form is set up to use the \code{axftn} routine that we have 
already shown is efficient. We could also use the
implicit matrix-vector product \code{axmolerfast}.

It seems overkill to show the RQ computation time for all versions
and matrices, so we will do the timing simply for a matrix of 
order 500.

\begin{knitrout}\scriptsize
\definecolor{shadecolor}{rgb}{0.969, 0.969, 0.969}\color{fgcolor}\begin{kframe}
\begin{alltt}
\hlkwd{require}\hlstd{(compiler)}
\hlstd{rqdirc}\hlkwb{<-}\hlkwd{cmpfun}\hlstd{(rqdir)}
\hlstd{ray1c}\hlkwb{<-}\hlkwd{cmpfun}\hlstd{(ray1)}
\hlstd{ray2c}\hlkwb{<-}\hlkwd{cmpfun}\hlstd{(ray2)}
\hlstd{ray3c}\hlkwb{<-}\hlkwd{cmpfun}\hlstd{(ray3)}
\hlkwd{dyn.load}\hlstd{(}\hlstr{"moler.so"}\hlstd{)}
  \hlstd{n}\hlkwb{<-}\hlnum{500}
  \hlstd{x}\hlkwb{<-}\hlkwd{runif}\hlstd{(n)} \hlcom{# generate a vector }
  \hlstd{AA}\hlkwb{<-}\hlkwd{molermat}\hlstd{(n)}
  \hlstd{tdi}\hlkwb{<-}\hlkwd{system.time}\hlstd{(rdi}\hlkwb{<-}\hlkwd{replicate}\hlstd{(}\hlnum{20}\hlstd{,}\hlkwd{rqdir}\hlstd{(x, AA))[}\hlnum{1}\hlstd{])[[}\hlnum{1}\hlstd{]]}
  \hlstd{tdc}\hlkwb{<-}\hlkwd{system.time}\hlstd{(}\hlkwd{replicate}\hlstd{(}\hlnum{20}\hlstd{,rdc}\hlkwb{<-}\hlkwd{rqdirc}\hlstd{(x, AA))[}\hlnum{1}\hlstd{])[[}\hlnum{1}\hlstd{]]}
  \hlkwd{cat}\hlstd{(}\hlstr{"Direct algorithm: interpreted="}\hlstd{,tdi,}\hlstr{"   byte-compiled="}\hlstd{,tdc,}\hlstr{"\textbackslash{}n"}\hlstd{)}
\end{alltt}
\begin{verbatim}
## Direct algorithm: interpreted= 1.952    byte-compiled= 1.944
\end{verbatim}
\begin{alltt}
  \hlstd{t1i}\hlkwb{<-}\hlkwd{system.time}\hlstd{(}\hlkwd{replicate}\hlstd{(}\hlnum{20}\hlstd{,r1i}\hlkwb{<-}\hlkwd{ray1}\hlstd{(x, AA))[}\hlnum{1}\hlstd{])[[}\hlnum{1}\hlstd{]]}
  \hlstd{t1c}\hlkwb{<-}\hlkwd{system.time}\hlstd{(}\hlkwd{replicate}\hlstd{(}\hlnum{20}\hlstd{,r1c}\hlkwb{<-}\hlkwd{ray1c}\hlstd{(x, AA))[}\hlnum{1}\hlstd{])[[}\hlnum{1}\hlstd{]]}
  \hlkwd{cat}\hlstd{(}\hlstr{"ray1: mat-mult algorithm: interpreted="}\hlstd{,t1i,}\hlstr{"   byte-compiled="}\hlstd{,t1c,}\hlstr{"\textbackslash{}n"}\hlstd{)}
\end{alltt}
\begin{verbatim}
## ray1: mat-mult algorithm: interpreted= 0.028    byte-compiled= 0.024
\end{verbatim}
\begin{alltt}
  \hlstd{t2i}\hlkwb{<-}\hlkwd{system.time}\hlstd{(}\hlkwd{replicate}\hlstd{(}\hlnum{20}\hlstd{,r2i}\hlkwb{<-}\hlkwd{ray2}\hlstd{(x, AA))[}\hlnum{1}\hlstd{])[[}\hlnum{1}\hlstd{]]}
  \hlstd{t2c}\hlkwb{<-}\hlkwd{system.time}\hlstd{(}\hlkwd{replicate}\hlstd{(}\hlnum{20}\hlstd{,r2c}\hlkwb{<-}\hlkwd{ray2c}\hlstd{(x, AA))[}\hlnum{1}\hlstd{])[[}\hlnum{1}\hlstd{]]}
  \hlkwd{cat}\hlstd{(}\hlstr{"ray2: crossprod algorithm: interpreted="}\hlstd{,t2i,}\hlstr{"   byte-compiled="}\hlstd{,t2c,}\hlstr{"\textbackslash{}n"}\hlstd{)}
\end{alltt}
\begin{verbatim}
## ray2: crossprod algorithm: interpreted= 0.032    byte-compiled= 0.02
\end{verbatim}
\begin{alltt}
  \hlstd{t3fi}\hlkwb{<-}\hlkwd{system.time}\hlstd{(}\hlkwd{replicate}\hlstd{(}\hlnum{20}\hlstd{,r3i}\hlkwb{<-}\hlkwd{ray3}\hlstd{(x, AA,} \hlkwc{ax}\hlstd{=axftn))[}\hlnum{1}\hlstd{])[[}\hlnum{1}\hlstd{]]}
  \hlstd{t3fc}\hlkwb{<-}\hlkwd{system.time}\hlstd{(}\hlkwd{replicate}\hlstd{(}\hlnum{20}\hlstd{,r3i}\hlkwb{<-}\hlkwd{ray3c}\hlstd{(x, AA,} \hlkwc{ax}\hlstd{=axftnc))[}\hlnum{1}\hlstd{])[[}\hlnum{1}\hlstd{]]}
  \hlkwd{cat}\hlstd{(}\hlstr{"ray3: ax Fortran + crossprod: interpreted="}\hlstd{,t3fi,}\hlstr{"   byte-compiled="}\hlstd{,t3fc,}\hlstr{"\textbackslash{}n"}\hlstd{)}
\end{alltt}
\begin{verbatim}
## ray3: ax Fortran + crossprod: interpreted= 0.032    byte-compiled= 0.028
\end{verbatim}
\begin{alltt}
  \hlstd{t3ri}\hlkwb{<-}\hlkwd{system.time}\hlstd{(}\hlkwd{replicate}\hlstd{(}\hlnum{20}\hlstd{,r3i}\hlkwb{<-}\hlkwd{ray3}\hlstd{(x, AA,} \hlkwc{ax}\hlstd{=axmolerfast))[}\hlnum{1}\hlstd{])[[}\hlnum{1}\hlstd{]]}
  \hlstd{t3rc}\hlkwb{<-}\hlkwd{system.time}\hlstd{(}\hlkwd{replicate}\hlstd{(}\hlnum{20}\hlstd{,r3i}\hlkwb{<-}\hlkwd{ray3c}\hlstd{(x, AA,} \hlkwc{ax}\hlstd{=axmfc))[}\hlnum{1}\hlstd{])[[}\hlnum{1}\hlstd{]]}
  \hlkwd{cat}\hlstd{(}\hlstr{"ray3: ax fast R implicit + crossprod: interpreted="}\hlstd{,t3ri,}\hlstr{"   byte-compiled="}\hlstd{,t3rc,}\hlstr{"\textbackslash{}n"}\hlstd{)}
\end{alltt}
\begin{verbatim}
## ray3: ax fast R implicit + crossprod: interpreted= 0.484    byte-compiled= 0.484
\end{verbatim}
\end{kframe}
\end{knitrout}

Here we see that the use of the \code{crossprod} in \code{ray2} is 
very fast, and this is interpreted code. Once again, we note that all 
timings except those for \code{ray3} should have some adjustment for
the building of the matrix. If storage is an issue, then \code{ray3},
which uses the implicit matrix-vector product in Fortran, is the 
approach of choice. My own preference would be to use this option
if the Fortran matrix-vector product subroutine is already available
for the matrix required. I would not, however, generally choose to
write the Fortran subroutine for a "new" problem matrix. 


\section{Solution by \code{spg}}

To actually solve the eigensolution problem we will first use the 
projected gradient method \code{spg} from \pkg{BB}. We repeat the 
RQ function so that it is clear which routine we are using.




\begin{knitrout}\scriptsize
\definecolor{shadecolor}{rgb}{0.969, 0.969, 0.969}\color{fgcolor}\begin{kframe}
\begin{alltt}
\hlstd{testsol} \hlkwb{<-} \hlkwa{function}\hlstd{(}\hlkwc{A}\hlstd{,}\hlkwc{eval}\hlstd{,} \hlkwc{evec}\hlstd{)\{} \hlcom{# test a trial eigensolution}
     \hlstd{cmpval} \hlkwb{<-} \hlkwd{length}\hlstd{(evec)}\hlopt{*}\hlnum{1e-5}
     \hlstd{tvec} \hlkwb{<-} \hlstd{A} \hlopt \hlstd{evec} \hlopt{-} \hlstd{eval}\hlopt{*}\hlstd{evec}
     \hlstd{etest} \hlkwb{<-} \hlkwd{max}\hlstd{(}\hlkwd{abs}\hlstd{(tvec))}
     \hlstd{ntest} \hlkwb{<-} \hlkwd{abs}\hlstd{(}\hlnum{1}\hlopt{-}\hlkwd{as.numeric}\hlstd{(}\hlkwd{crossprod}\hlstd{(evec)))}
     \hlkwd{cat}\hlstd{(}\hlstr{"Eigenvalue solution test:"}\hlstd{, etest ,}\hlstr{"   Normtest:"}\hlstd{, ntest ,}\hlstr{"\textbackslash{}n"}\hlstd{)}
     \hlkwa{if} \hlstd{((ntest} \hlopt{>} \hlstd{cmpval)} \hlopt{||} \hlstd{(etest} \hlopt{>} \hlstd{cmpval))} \hlstr{"FAIL"} \hlkwa{else} \hlstr{"OK"}
\hlstd{\}}
\hlstd{rqt}\hlkwb{<-}\hlkwa{function}\hlstd{(}\hlkwc{x}\hlstd{,} \hlkwc{AA}\hlstd{)\{}
    \hlstd{rq}\hlkwb{<-}\hlkwd{as.numeric}\hlstd{(}\hlkwd{crossprod}\hlstd{(x,} \hlkwd{crossprod}\hlstd{(AA,x)))}\hlopt{/}\hlkwd{as.numeric}\hlstd{(}\hlkwd{crossprod}\hlstd{(x))}
\hlstd{\}}
\hlstd{proj}\hlkwb{<-}\hlkwa{function}\hlstd{(}\hlkwc{x}\hlstd{) \{ x}\hlopt{/}\hlkwd{sqrt}\hlstd{(}\hlkwd{crossprod}\hlstd{(x)) \}}
\hlkwd{require}\hlstd{(BB)}
\end{alltt}


{\ttfamily\noindent\itshape\color{messagecolor}{\#\# Loading required package: BB}}

{\ttfamily\noindent\color{warningcolor}{\#\# Warning in library(package, lib.loc = lib.loc, character.only = TRUE, logical.return = TRUE, : there is no package called 'BB'}}\begin{alltt}
\hlstd{n}\hlkwb{<-}\hlnum{100}
\hlstd{x}\hlkwb{<-}\hlkwd{rep}\hlstd{(}\hlnum{1}\hlstd{,n)}
\hlstd{AA}\hlkwb{<-}\hlkwd{molermat}\hlstd{(n)}
\hlstd{tevs} \hlkwb{<-} \hlkwd{system.time}\hlstd{(evs}\hlkwb{<-}\hlkwd{eigen}\hlstd{(AA))[[}\hlnum{1}\hlstd{]]}
\hlkwd{cat}\hlstd{(}\hlstr{"Time to compute full eigensystem = "}\hlstd{, tevs,} \hlstr{"\textbackslash{}n"}\hlstd{)}
\end{alltt}
\begin{verbatim}
## Time to compute full eigensystem =  0.008
\end{verbatim}
\begin{alltt}
\hlstd{tmin}\hlkwb{<-}\hlkwd{system.time}\hlstd{(amin}\hlkwb{<-}\hlkwd{spg}\hlstd{(x,} \hlkwc{fn}\hlstd{=rqt,} \hlkwc{project}\hlstd{=proj,} \hlkwc{control}\hlstd{=}\hlkwd{list}\hlstd{(}\hlkwc{trace}\hlstd{=}\hlnum{FALSE}\hlstd{),} \hlkwc{AA}\hlstd{=AA))[[}\hlnum{1}\hlstd{]]}
\end{alltt}


{\ttfamily\noindent\bfseries\color{errorcolor}{\#\# Error in spg(x, fn = rqt, project = proj, control = list(trace = FALSE), : could not find function "{}spg"{}}}

{\ttfamily\noindent\itshape\color{messagecolor}{\#\# Timing stopped at: 0 0 0.001}}\begin{alltt}
\hlcom{#amin}
\hlstd{tmax}\hlkwb{<-}\hlkwd{system.time}\hlstd{(amax}\hlkwb{<-}\hlkwd{spg}\hlstd{(x,} \hlkwc{fn}\hlstd{=rqt,} \hlkwc{project}\hlstd{=proj,} \hlkwc{control}\hlstd{=}\hlkwd{list}\hlstd{(}\hlkwc{trace}\hlstd{=}\hlnum{FALSE}\hlstd{),} \hlkwc{AA}\hlstd{=}\hlopt{-}\hlstd{AA))[[}\hlnum{1}\hlstd{]]}
\end{alltt}


{\ttfamily\noindent\bfseries\color{errorcolor}{\#\# Error in spg(x, fn = rqt, project = proj, control = list(trace = FALSE), : could not find function "{}spg"{}}}

{\ttfamily\noindent\itshape\color{messagecolor}{\#\# Timing stopped at: 0 0 0}}\begin{alltt}
\hlcom{#amax}
\hlstd{evalmax}\hlkwb{<-}\hlstd{evs}\hlopt{$}\hlstd{values[}\hlnum{1}\hlstd{]}
\hlstd{evecmax}\hlkwb{<-}\hlstd{evs}\hlopt{$}\hlstd{vectors[,}\hlnum{1}\hlstd{]}
\hlstd{evecmax}\hlkwb{<-}\hlkwd{sign}\hlstd{(evecmax[}\hlnum{1}\hlstd{])}\hlopt{*}\hlstd{evecmax}\hlopt{/}\hlkwd{sqrt}\hlstd{(}\hlkwd{as.numeric}\hlstd{(}\hlkwd{crossprod}\hlstd{(evecmax)))}
\hlstd{emax}\hlkwb{<-}\hlkwd{list}\hlstd{(}\hlkwc{evalmax}\hlstd{=evalmax,} \hlkwc{evecmax}\hlstd{=evecmax)}
\hlkwd{save}\hlstd{(emax,} \hlkwc{file}\hlstd{=}\hlstr{"temax.Rdata"}\hlstd{)}
\hlstd{evalmin}\hlkwb{<-}\hlstd{evs}\hlopt{$}\hlstd{values[n]}
\hlstd{evecmin}\hlkwb{<-}\hlstd{evs}\hlopt{$}\hlstd{vectors[,n]}
\hlstd{evecmin}\hlkwb{<-}\hlkwd{sign}\hlstd{(evecmin[}\hlnum{1}\hlstd{])}\hlopt{*}\hlstd{evecmin}\hlopt{/}\hlkwd{sqrt}\hlstd{(}\hlkwd{as.numeric}\hlstd{(}\hlkwd{crossprod}\hlstd{(evecmin)))}
\hlstd{avecmax}\hlkwb{<-}\hlstd{amax}\hlopt{$}\hlstd{par}
\end{alltt}


{\ttfamily\noindent\bfseries\color{errorcolor}{\#\# Error in eval(expr, envir, enclos): object 'amax' not found}}\begin{alltt}
\hlstd{avecmin}\hlkwb{<-}\hlstd{amin}\hlopt{$}\hlstd{par}
\end{alltt}


{\ttfamily\noindent\bfseries\color{errorcolor}{\#\# Error in eval(expr, envir, enclos): object 'amin' not found}}\begin{alltt}
\hlstd{avecmax}\hlkwb{<-}\hlkwd{sign}\hlstd{(avecmax[}\hlnum{1}\hlstd{])}\hlopt{*}\hlstd{avecmax}\hlopt{/}\hlkwd{sqrt}\hlstd{(}\hlkwd{as.numeric}\hlstd{(}\hlkwd{crossprod}\hlstd{(avecmax)))}
\end{alltt}


{\ttfamily\noindent\bfseries\color{errorcolor}{\#\# Error in eval(expr, envir, enclos): object 'avecmax' not found}}\begin{alltt}
\hlstd{avecmin}\hlkwb{<-}\hlkwd{sign}\hlstd{(avecmin[}\hlnum{1}\hlstd{])}\hlopt{*}\hlstd{avecmin}\hlopt{/}\hlkwd{sqrt}\hlstd{(}\hlkwd{as.numeric}\hlstd{(}\hlkwd{crossprod}\hlstd{(avecmin)))}
\end{alltt}


{\ttfamily\noindent\bfseries\color{errorcolor}{\#\# Error in eval(expr, envir, enclos): object 'avecmin' not found}}\begin{alltt}
\hlkwd{cat}\hlstd{(}\hlstr{"minimal eigensolution: Value="}\hlstd{,amin}\hlopt{$}\hlstd{value,}\hlstr{"in time "}\hlstd{,tmin,}\hlstr{"\textbackslash{}n"}\hlstd{)}
\end{alltt}


{\ttfamily\noindent\bfseries\color{errorcolor}{\#\# Error in cat("{}minimal eigensolution: Value="{}, amin\$value, "{}in time "{}, : object 'amin' not found}}\begin{alltt}
\hlkwd{cat}\hlstd{(}\hlstr{"Eigenvalue - result from eigen="}\hlstd{,amin}\hlopt{$}\hlstd{value}\hlopt{-}\hlstd{evalmin,}\hlstr{"  vector max(abs(diff))="}\hlstd{,}
      \hlkwd{max}\hlstd{(}\hlkwd{abs}\hlstd{(avecmin}\hlopt{-}\hlstd{evecmin)),}\hlstr{"\textbackslash{}n\textbackslash{}n"}\hlstd{)}
\end{alltt}


{\ttfamily\noindent\bfseries\color{errorcolor}{\#\# Error in cat("{}Eigenvalue - result from eigen="{}, amin\$value - evalmin, : object 'amin' not found}}\begin{alltt}
\hlkwd{print}\hlstd{(}\hlkwd{testsol}\hlstd{(AA, amin}\hlopt{$}\hlstd{value, avecmin))}
\end{alltt}


{\ttfamily\noindent\bfseries\color{errorcolor}{\#\# Error in testsol(AA, amin\$value, avecmin): object 'avecmin' not found}}\begin{alltt}
\hlkwd{cat}\hlstd{(}\hlstr{"maximal eigensolution: Value="}\hlstd{,}\hlopt{-}\hlstd{amax}\hlopt{$}\hlstd{value,}\hlstr{"in time "}\hlstd{,tmax,}\hlstr{"\textbackslash{}n"}\hlstd{)}
\end{alltt}


{\ttfamily\noindent\bfseries\color{errorcolor}{\#\# Error in cat("{}maximal eigensolution: Value="{}, -amax\$value, "{}in time "{}, : object 'amax' not found}}\begin{alltt}
\hlkwd{cat}\hlstd{(}\hlstr{"Eigenvalue - result from eigen="}\hlstd{,}\hlopt{-}\hlstd{amax}\hlopt{$}\hlstd{value}\hlopt{-}\hlstd{evalmax,}\hlstr{"  vector max(abs(diff))="}\hlstd{,}
      \hlkwd{max}\hlstd{(}\hlkwd{abs}\hlstd{(avecmax}\hlopt{-}\hlstd{evecmax)),}\hlstr{"\textbackslash{}n\textbackslash{}n"}\hlstd{)}
\end{alltt}


{\ttfamily\noindent\bfseries\color{errorcolor}{\#\# Error in cat("{}Eigenvalue - result from eigen="{}, -amax\$value - evalmax, : object 'amax' not found}}\begin{alltt}
\hlkwd{print}\hlstd{(}\hlkwd{testsol}\hlstd{(AA,} \hlopt{-}\hlstd{amax}\hlopt{$}\hlstd{value, avecmax))}
\end{alltt}


{\ttfamily\noindent\bfseries\color{errorcolor}{\#\# Error in testsol(AA, -amax\$value, avecmax): object 'avecmax' not found}}\end{kframe}
\end{knitrout}

\begin{knitrout}\scriptsize
\definecolor{shadecolor}{rgb}{0.969, 0.969, 0.969}\color{fgcolor}\begin{kframe}


{\ttfamily\noindent\itshape\color{messagecolor}{\#\# Loading required package: BB}}

{\ttfamily\noindent\color{warningcolor}{\#\# Warning in library(package, lib.loc = lib.loc, character.only = TRUE, logical.return = TRUE, : there is no package called 'BB'}}

{\ttfamily\noindent\bfseries\color{errorcolor}{\#\# Error in cmpfun(spg): object 'spg' not found}}

{\ttfamily\noindent\bfseries\color{errorcolor}{\#\# Error in spg(x, fn = rqt, project = proj, control = list(trace = FALSE), : could not find function "{}spg"{}}}

{\ttfamily\noindent\itshape\color{messagecolor}{\#\# Timing stopped at: 0 0 0}}\begin{verbatim}
##     n spgrqt spgcrqtcaxc tbldc
## 1  50     NA          NA    NA
## 2  NA     NA          NA    NA
## 3  NA     NA          NA    NA
## 4  NA     NA          NA    NA
## 5  NA     NA          NA    NA
## 6  NA     NA          NA    NA
## 7  NA     NA          NA    NA
## 8  NA     NA          NA    NA
## 9  NA     NA          NA    NA
## 10 NA     NA          NA    NA
\end{verbatim}
\end{kframe}
\end{knitrout}


\section{Solution by other optimizers}

We can try other optimizers, but we must note that unlike \code{spg} they 
do not take account explicitly of the scaling. However, we can build in a transformation, 
since our function is always the same for all sets of parameters scaled by the
square root of the parameter inner product. The function \code{nobj} forms the
quadratic form that is the numerator of the Rayleigh Quotient using the more 
efficient \code{crossprod()} function

\code{    rq<- as.numeric(crossprod(y, crossprod(AA,y))) }

but we first form

\code{    y<-x/sqrt(as.numeric(crossprod(x))) }

to scale the parameters. 

Since we are running a number of gradient-based optimizers in the wrapper
\code{optimx}, we have reduced the matrix sizes and numbers.


\begin{knitrout}\scriptsize
\definecolor{shadecolor}{rgb}{0.969, 0.969, 0.969}\color{fgcolor}\begin{kframe}
\begin{alltt}
\hlstd{nobj}\hlkwb{<-}\hlkwa{function}\hlstd{(}\hlkwc{x}\hlstd{,} \hlkwc{AA}\hlstd{=}\hlopt{-}\hlstd{AA)\{}
   \hlstd{y}\hlkwb{<-}\hlstd{x}\hlopt{/}\hlkwd{sqrt}\hlstd{(}\hlkwd{as.numeric}\hlstd{(}\hlkwd{crossprod}\hlstd{(x)))} \hlcom{# prescale}
   \hlstd{rq}\hlkwb{<-} \hlkwd{as.numeric}\hlstd{(}\hlkwd{crossprod}\hlstd{(y,} \hlkwd{crossprod}\hlstd{(AA,y)))}
\hlstd{\}}

\hlstd{ngrobj}\hlkwb{<-}\hlkwa{function}\hlstd{(}\hlkwc{x}\hlstd{,} \hlkwc{AA}\hlstd{=}\hlopt{-}\hlstd{AA)\{}
   \hlstd{y}\hlkwb{<-}\hlstd{x}\hlopt{/}\hlkwd{sqrt}\hlstd{(}\hlkwd{as.numeric}\hlstd{(}\hlkwd{crossprod}\hlstd{(x)))}
   \hlstd{n}\hlkwb{<-}\hlkwd{length}\hlstd{(x)}
   \hlstd{dd}\hlkwb{<-}\hlkwd{sqrt}\hlstd{(}\hlkwd{as.numeric}\hlstd{(}\hlkwd{crossprod}\hlstd{(x)))}
   \hlstd{T1}\hlkwb{<-}\hlkwd{diag}\hlstd{(}\hlkwd{rep}\hlstd{(}\hlnum{1}\hlstd{,n))}\hlopt{/}\hlstd{dd}
   \hlstd{T2}\hlkwb{<-} \hlstd{x}\hlopt\hlstd{x}\hlopt{/}\hlstd{(dd}\hlopt{*}\hlstd{dd}\hlopt{*}\hlstd{dd)}
   \hlstd{gt}\hlkwb{<-}\hlstd{T1}\hlopt{-}\hlstd{T2}
   \hlstd{gy}\hlkwb{<-} \hlkwd{as.vector}\hlstd{(}\hlnum{2.}\hlopt{*}\hlkwd{crossprod}\hlstd{(AA,y))}
   \hlstd{gg}\hlkwb{<-}\hlkwd{as.numeric}\hlstd{(}\hlkwd{crossprod}\hlstd{(gy, gt))}
\hlstd{\}}
\hlkwd{require}\hlstd{(optimrx)}
\end{alltt}


{\ttfamily\noindent\itshape\color{messagecolor}{\#\# Loading required package: optimrx}}

{\ttfamily\noindent\color{warningcolor}{\#\# Warning in library(package, lib.loc = lib.loc, character.only = TRUE, logical.return = TRUE, : there is no package called 'optimrx'}}\begin{alltt}
\hlcom{# mset<-c("L-BFGS-B", "BFGS", "CG", "spg", "ucminf", "nlm", "nlminb", "Rvmmin", "Rcgmin")}
\hlstd{mset}\hlkwb{<-}\hlstr{"Rcgmin"}
\hlstd{nmax}\hlkwb{<-}\hlnum{5}
\hlkwa{for} \hlstd{(ni} \hlkwa{in} \hlnum{1}\hlopt{:}\hlstd{nmax)\{}
  \hlstd{n}\hlkwb{<-}\hlnum{20}\hlopt{*}\hlstd{ni}
  \hlstd{x}\hlkwb{<-}\hlkwd{runif}\hlstd{(n)} \hlcom{# generate a vector }
\hlcom{#  AA<-molerc(n) # make sure defined}
  \hlstd{AA}\hlkwb{<-}\hlkwd{molermat}\hlstd{(n)}
  \hlstd{aall}\hlkwb{<-}\hlkwd{opm}\hlstd{(x,} \hlkwc{fn}\hlstd{=nobj,} \hlkwc{gr}\hlstd{=ngrobj,} \hlkwc{method}\hlstd{=mset,} \hlkwc{AA}\hlstd{=}\hlopt{-}\hlstd{AA,}
     \hlkwc{control}\hlstd{=}\hlkwd{list}\hlstd{(}\hlkwc{starttests}\hlstd{=}\hlnum{FALSE}\hlstd{,}  \hlkwc{dowarn}\hlstd{=}\hlnum{FALSE}\hlstd{))}
  \hlkwd{print}\hlstd{(}\hlkwd{summary}\hlstd{(aall,} \hlkwc{order}\hlstd{=value,} \hlkwc{par.select}\hlstd{=}\hlnum{1}\hlopt{:}\hlnum{3}\hlstd{))}
  \hlkwd{cat}\hlstd{(}\hlstr{"Above for n="}\hlstd{,n,}\hlstr{" \textbackslash{}n"}\hlstd{)}
\hlstd{\}}
\end{alltt}


{\ttfamily\noindent\bfseries\color{errorcolor}{\#\# Error in opm(x, fn = nobj, gr = ngrobj, method = mset, AA = -AA, control = list(starttests = FALSE, : could not find function "{}opm"{}}}\end{kframe}
\end{knitrout}

The timings for these matrices of order 20 to 100 are likely too short to be very
reliable in detail, but do show that the RQ problem using the scaling transformation
and with an analytic gradient can be solved very quickly, especially by the limited
memory methods such as L-BFGS-B and Rcgmin. Below we use the latter (in its 
unconstrained implementation) to show the times over different matrix sizes.

\begin{knitrout}\scriptsize
\definecolor{shadecolor}{rgb}{0.969, 0.969, 0.969}\color{fgcolor}\begin{kframe}
\begin{alltt}
\hlkwd{library}\hlstd{(optimrx)}
\end{alltt}


{\ttfamily\noindent\bfseries\color{errorcolor}{\#\# Error in library(optimrx): there is no package called 'optimrx'}}\begin{alltt}
\hlstd{ctable}\hlkwb{<-}\hlkwd{matrix}\hlstd{(}\hlnum{NA}\hlstd{,} \hlkwc{nrow}\hlstd{=}\hlnum{10}\hlstd{,} \hlkwc{ncol}\hlstd{=}\hlnum{2}\hlstd{)}
\hlstd{nmax}\hlkwb{<-}\hlnum{10}
\hlkwa{for} \hlstd{(ni} \hlkwa{in} \hlnum{1}\hlopt{:}\hlstd{nmax)\{}
  \hlstd{n}\hlkwb{<-}\hlnum{50}\hlopt{*}\hlstd{ni}
  \hlstd{x}\hlkwb{<-}\hlkwd{runif}\hlstd{(n)} \hlcom{# generate a vector }
\hlcom{#  AA<-molerc(n) # make sure defined}
  \hlstd{AA}\hlkwb{<-}\hlkwd{molermat}\hlstd{(n)}
  \hlstd{tcgu}\hlkwb{<-}\hlkwd{system.time}\hlstd{(arcgu}\hlkwb{<-}\hlkwd{optimr}\hlstd{(x,} \hlkwc{fn}\hlstd{=nobj,} \hlkwc{gr}\hlstd{=ngrobj,} \hlkwc{method}\hlstd{=}\hlstr{"Rcgmin"}\hlstd{,} \hlkwc{AA}\hlstd{=}\hlopt{-}\hlstd{AA))[[}\hlnum{1}\hlstd{]]}
  \hlstd{ctable[[ni,}\hlnum{1}\hlstd{]]} \hlkwb{<-} \hlstd{n}
  \hlstd{ctable[[ni,}\hlnum{2}\hlstd{]]} \hlkwb{<-} \hlstd{tcgu}
\hlstd{\}}
\end{alltt}


{\ttfamily\noindent\bfseries\color{errorcolor}{\#\# Error in optimr(x, fn = nobj, gr = ngrobj, method = "{}Rcgmin"{}, AA = -AA): could not find function "{}optimr"{}}}

{\ttfamily\noindent\itshape\color{messagecolor}{\#\# Timing stopped at: 0.004 0 0}}\begin{alltt}
\hlstd{cgtime}\hlkwb{<-}\hlkwd{data.frame}\hlstd{(}\hlkwc{n}\hlstd{=ctable[,}\hlnum{1}\hlstd{],} \hlkwc{tRcgminu}\hlstd{=ctable[,}\hlnum{2}\hlstd{])}
\hlkwd{print}\hlstd{(cgtime)}
\end{alltt}
\begin{verbatim}
##     n tRcgminu
## 1  NA       NA
## 2  NA       NA
## 3  NA       NA
## 4  NA       NA
## 5  NA       NA
## 6  NA       NA
## 7  NA       NA
## 8  NA       NA
## 9  NA       NA
## 10 NA       NA
\end{verbatim}
\end{kframe}
\end{knitrout}

\section{A specialized minimizer - Geradin's method}

For comparison, let us try the \cite{Geradin71} routine (Appendix 1) as implemented in \R by one of 
us (JN). This is a specialized conjugate gradient minimization routine for eigenvalue problems.



\begin{knitrout}\scriptsize
\definecolor{shadecolor}{rgb}{0.969, 0.969, 0.969}\color{fgcolor}\begin{kframe}
\begin{alltt}
\hlkwd{cat}\hlstd{(}\hlstr{"Test geradin with explicit matrix multiplication\textbackslash{}n"}\hlstd{)}
\end{alltt}
\begin{verbatim}
## Test geradin with explicit matrix multiplication
\end{verbatim}
\begin{alltt}
\hlstd{n}\hlkwb{<-}\hlnum{10}
\hlstd{AA}\hlkwb{<-}\hlkwd{molermat}\hlstd{(n)}
\hlstd{BB}\hlkwb{=}\hlkwd{diag}\hlstd{(}\hlkwd{rep}\hlstd{(}\hlnum{1}\hlstd{,n))}
\hlstd{x}\hlkwb{<-}\hlkwd{runif}\hlstd{(n)}
\hlstd{tg}\hlkwb{<-}\hlkwd{system.time}\hlstd{(ag}\hlkwb{<-}\hlkwd{geradin}\hlstd{(x, ax, bx,} \hlkwc{AA}\hlstd{=AA,} \hlkwc{BB}\hlstd{=BB,}
   \hlkwc{control}\hlstd{=}\hlkwd{list}\hlstd{(}\hlkwc{trace}\hlstd{=}\hlnum{FALSE}\hlstd{)))[[}\hlnum{1}\hlstd{]]}
\hlkwd{cat}\hlstd{(}\hlstr{"Minimal eigensolution\textbackslash{}n"}\hlstd{)}
\end{alltt}
\begin{verbatim}
## Minimal eigensolution
\end{verbatim}
\begin{alltt}
\hlkwd{print}\hlstd{(ag)}
\end{alltt}
\begin{verbatim}
## $x
##  [1] -706161.664 -353082.348 -176544.962  -88280.436  -44156.317
##  [6]  -22110.452  -11119.860   -5689.222   -3103.207   -2068.801
## 
## $RQ
## [1] 8.582807e-06
## 
## $ipr
## [1] 51
## 
## $msg
## [1] "Small gradient -- done"
\end{verbatim}
\begin{alltt}
\hlkwd{cat}\hlstd{(}\hlstr{"Geradin time="}\hlstd{,tg,}\hlstr{"\textbackslash{}n"}\hlstd{)}
\end{alltt}
\begin{verbatim}
## Geradin time= 0.2
\end{verbatim}
\begin{alltt}
\hlstd{tgn}\hlkwb{<-}\hlkwd{system.time}\hlstd{(agn}\hlkwb{<-}\hlkwd{geradin}\hlstd{(x, ax, bx,} \hlkwc{AA}\hlstd{=}\hlopt{-}\hlstd{AA,} \hlkwc{BB}\hlstd{=BB,}
   \hlkwc{control}\hlstd{=}\hlkwd{list}\hlstd{(}\hlkwc{trace}\hlstd{=}\hlnum{FALSE}\hlstd{)))[[}\hlnum{1}\hlstd{]]}
\hlkwd{cat}\hlstd{(}\hlstr{"Maximal eigensolution (negative matrix)\textbackslash{}n"}\hlstd{)}
\end{alltt}
\begin{verbatim}
## Maximal eigensolution (negative matrix)
\end{verbatim}
\begin{alltt}
\hlkwd{print}\hlstd{(agn)}
\end{alltt}
\begin{verbatim}
## $x
##  [1]  10139500929   -342726276 -10813395547 -20918690433 -30317033945
##  [6] -38690828537 -45756996144 -51276766405 -55063595421 -56989550735
## 
## $RQ
## [1] -31.58981
## 
## $ipr
## [1] 37
## 
## $msg
## [1] "Small gradient -- done"
\end{verbatim}
\begin{alltt}
\hlkwd{cat}\hlstd{(}\hlstr{"Geradin time="}\hlstd{,tgn,}\hlstr{"\textbackslash{}n"}\hlstd{)}
\end{alltt}
\begin{verbatim}
## Geradin time= 0.004
\end{verbatim}
\end{kframe}
\end{knitrout}

Let us time this routine with different matrix vector approaches.

\begin{knitrout}\scriptsize
\definecolor{shadecolor}{rgb}{0.969, 0.969, 0.969}\color{fgcolor}\begin{kframe}
\begin{alltt}
\hlstd{naximp}\hlkwb{<-}\hlkwa{function}\hlstd{(}\hlkwc{x}\hlstd{,} \hlkwc{A}\hlstd{=}\hlnum{1}\hlstd{)\{} \hlcom{# implicit moler A*x}
   \hlstd{n}\hlkwb{<-}\hlkwd{length}\hlstd{(x)}
   \hlstd{y}\hlkwb{<-}\hlkwd{rep}\hlstd{(}\hlnum{0}\hlstd{,n)}
   \hlkwa{for} \hlstd{(i} \hlkwa{in} \hlnum{1}\hlopt{:}\hlstd{n)\{}
      \hlstd{tt}\hlkwb{<-}\hlnum{0.}
      \hlkwa{for} \hlstd{(j} \hlkwa{in} \hlnum{1}\hlopt{:}\hlstd{n) \{}
          \hlkwa{if} \hlstd{(i} \hlopt{==} \hlstd{j) tt}\hlkwb{<-}\hlstd{tt}\hlopt{+}\hlstd{i}\hlopt{*}\hlstd{x[i]}
          \hlkwa{else} \hlstd{tt}\hlkwb{<-}\hlstd{tt}\hlopt{+}\hlstd{(}\hlkwd{min}\hlstd{(i,j)} \hlopt{-} \hlnum{2}\hlstd{)}\hlopt{*}\hlstd{x[j]}
      \hlstd{\}}
      \hlstd{y[i]}\hlkwb{<-} \hlopt{-}\hlstd{tt} \hlcom{# include negative sign}
   \hlstd{\}}
   \hlstd{y}
\hlstd{\}}

\hlkwd{dyn.load}\hlstd{(}\hlstr{"moler.so"}\hlstd{)}
\hlkwd{cat}\hlstd{(}\hlstr{"Is the mat multiply loaded? "}\hlstd{,}\hlkwd{is.loaded}\hlstd{(}\hlstr{"moler"}\hlstd{),}\hlstr{"\textbackslash{}n"}\hlstd{)}
\end{alltt}
\begin{verbatim}
## Is the mat multiply loaded?  TRUE
\end{verbatim}
\begin{alltt}
\hlstd{naxftn}\hlkwb{<-}\hlkwa{function}\hlstd{(}\hlkwc{x}\hlstd{,} \hlkwc{A}\hlstd{) \{} \hlcom{# ignore second argument}
   \hlstd{n}\hlkwb{<-}\hlkwd{length}\hlstd{(x)} \hlcom{# could speed up by having this passed}
   \hlstd{vout}\hlkwb{<-}\hlkwd{rep}\hlstd{(}\hlnum{0}\hlstd{,n)} \hlcom{# purely for storage}
   \hlstd{res}\hlkwb{<-}\hlstd{(}\hlopt{-}\hlnum{1}\hlstd{)}\hlopt{*}\hlstd{(}\hlkwd{.Fortran}\hlstd{(}\hlstr{"moler"}\hlstd{,} \hlkwc{n}\hlstd{=}\hlkwd{as.integer}\hlstd{(n),} \hlkwc{x}\hlstd{=}\hlkwd{as.double}\hlstd{(x),} \hlkwc{vout}\hlstd{=}\hlkwd{as.double}\hlstd{(vout)))}\hlopt{$}\hlstd{vout}
\hlstd{\}}

\hlkwd{require}\hlstd{(compiler)}
\hlstd{naxftnc}\hlkwb{<-}\hlkwd{cmpfun}\hlstd{(naxftn)}
\hlstd{naximpc}\hlkwb{<-}\hlkwd{cmpfun}\hlstd{(naximp)}

\hlcom{# require(microbenchmark)}
\hlstd{nmax}\hlkwb{<-}\hlnum{10}
\hlstd{gtable}\hlkwb{<-}\hlkwd{matrix}\hlstd{(}\hlnum{NA}\hlstd{,} \hlkwc{nrow}\hlstd{=nmax,} \hlkwc{ncol}\hlstd{=}\hlnum{6}\hlstd{)} \hlcom{# to hold results}
\hlcom{# loop over sizes}
\hlkwa{for} \hlstd{(ni} \hlkwa{in} \hlnum{1}\hlopt{:}\hlstd{nmax)\{}
  \hlstd{n}\hlkwb{<-}\hlnum{50}\hlopt{*}\hlstd{ni}
  \hlstd{x}\hlkwb{<-}\hlkwd{runif}\hlstd{(n)} \hlcom{# generate a vector }
  \hlstd{gtable[[ni,} \hlnum{1}\hlstd{]]}\hlkwb{<-}\hlstd{n}
  \hlstd{AA}\hlkwb{<-}\hlkwd{molermat}\hlstd{(n)}
  \hlstd{BB}\hlkwb{<-}\hlkwd{diag}\hlstd{(}\hlkwd{rep}\hlstd{(}\hlnum{1}\hlstd{,n))}
  \hlstd{tgax}\hlkwb{<-}\hlkwd{system.time}\hlstd{(ogax}\hlkwb{<-}\hlkwd{geradin}\hlstd{(x, ax, bx,} \hlkwc{AA}\hlstd{=}\hlopt{-}\hlstd{AA,} \hlkwc{BB}\hlstd{=BB,} \hlkwc{control}\hlstd{=}\hlkwd{list}\hlstd{(}\hlkwc{trace}\hlstd{=}\hlnum{FALSE}\hlstd{)))[[}\hlnum{1}\hlstd{]]}
  \hlstd{gtable[[ni,} \hlnum{2}\hlstd{]]}\hlkwb{<-}\hlstd{tgax}
  \hlstd{tgaximp}\hlkwb{<-}\hlkwd{system.time}\hlstd{(ogaximp}\hlkwb{<-}\hlkwd{geradin}\hlstd{(x, naximp, ident,} \hlkwc{AA}\hlstd{=}\hlnum{1}\hlstd{,} \hlkwc{BB}\hlstd{=}\hlnum{1}\hlstd{,} \hlkwc{control}\hlstd{=}\hlkwd{list}\hlstd{(}\hlkwc{trace}\hlstd{=}\hlnum{FALSE}\hlstd{)))[[}\hlnum{1}\hlstd{]]}
  \hlstd{gtable[[ni,} \hlnum{3}\hlstd{]]}\hlkwb{<-}\hlstd{tgaximp}
  \hlstd{tgaximpc}\hlkwb{<-}\hlkwd{system.time}\hlstd{(ogaximpc}\hlkwb{<-}\hlkwd{geradin}\hlstd{(x, naximpc, ident,} \hlkwc{AA}\hlstd{=}\hlnum{1}\hlstd{,} \hlkwc{BB}\hlstd{=}\hlnum{1}\hlstd{,} \hlkwc{control}\hlstd{=}\hlkwd{list}\hlstd{(}\hlkwc{trace}\hlstd{=}\hlnum{FALSE}\hlstd{)))[[}\hlnum{1}\hlstd{]]}
  \hlstd{gtable[[ni,} \hlnum{4}\hlstd{]]}\hlkwb{<-}\hlstd{tgaximpc}
  \hlstd{tgaxftn}\hlkwb{<-}\hlkwd{system.time}\hlstd{(ogaxftn}\hlkwb{<-}\hlkwd{geradin}\hlstd{(x, naxftn, ident,} \hlkwc{AA}\hlstd{=}\hlnum{1}\hlstd{,} \hlkwc{BB}\hlstd{=}\hlnum{1}\hlstd{,} \hlkwc{control}\hlstd{=}\hlkwd{list}\hlstd{(}\hlkwc{trace}\hlstd{=}\hlnum{FALSE}\hlstd{)))[[}\hlnum{1}\hlstd{]]}
  \hlstd{gtable[[ni,} \hlnum{5}\hlstd{]]}\hlkwb{<-}\hlstd{tgaxftn}
  \hlstd{tgaxftnc}\hlkwb{<-}\hlkwd{system.time}\hlstd{(ogaxftnc}\hlkwb{<-}\hlkwd{geradin}\hlstd{(x, naxftnc, ident,} \hlkwc{AA}\hlstd{=}\hlnum{1}\hlstd{,} \hlkwc{BB}\hlstd{=}\hlnum{1}\hlstd{,} \hlkwc{control}\hlstd{=}\hlkwd{list}\hlstd{(}\hlkwc{trace}\hlstd{=}\hlnum{FALSE}\hlstd{)))[[}\hlnum{1}\hlstd{]]}
  \hlstd{gtable[[ni,} \hlnum{6}\hlstd{]]}\hlkwb{<-}\hlstd{tgaxftnc}
\hlcom{#  cat(n,tgax, tgaximp, tgaximpc, tgaxftn, tgaxftnc,"\textbackslash{}n")}
\hlstd{\}}

\hlstd{gtym}\hlkwb{<-}\hlkwd{data.frame}\hlstd{(}\hlkwc{n}\hlstd{=gtable[,}\hlnum{1}\hlstd{],} \hlkwc{ax}\hlstd{=gtable[,}\hlnum{2}\hlstd{],} \hlkwc{aximp}\hlstd{=gtable[,}\hlnum{3}\hlstd{],}
  \hlkwc{aximpc}\hlstd{=gtable[,}\hlnum{4}\hlstd{],} \hlkwc{axftn}\hlstd{=gtable[,}\hlnum{5}\hlstd{],} \hlkwc{axftnc}\hlstd{=gtable[,}\hlnum{6}\hlstd{])}
\hlkwd{print}\hlstd{(gtym)}
\end{alltt}
\begin{verbatim}
##      n    ax aximp aximpc axftn axftnc
## 1   50 0.000 0.116  0.084 0.012  0.000
## 2  100 0.008 0.344  0.320 0.004  0.004
## 3  150 0.008 0.768  0.768 0.008  0.008
## 4  200 0.016 1.372  1.436 0.008  0.008
## 5  250 0.016 2.300  2.312 0.012  0.016
## 6  300 0.024 2.780  2.836 0.020  0.016
## 7  350 0.036 4.280  4.340 0.028  0.028
## 8  400 0.048 5.888  5.944 0.036  0.032
## 9  450 0.064 7.492  7.504 0.040  0.044
## 10 500 0.080 9.252  9.252 0.048  0.048
\end{verbatim}
\end{kframe}
\end{knitrout}

Let us check that the solution for \code{n = 100} by Geradin is consistent
with the answer via \code{eigen()}.

\begin{knitrout}\scriptsize
\definecolor{shadecolor}{rgb}{0.969, 0.969, 0.969}\color{fgcolor}\begin{kframe}
\begin{alltt}
\hlstd{n}\hlkwb{<-}\hlnum{100}
\hlstd{x}\hlkwb{<-}\hlkwd{runif}\hlstd{(n)}
\hlkwd{load}\hlstd{(}\hlstr{"temax.Rdata"}\hlstd{)}
\hlstd{evalmax}\hlkwb{<-}\hlstd{emax}\hlopt{$}\hlstd{evalmax}
\hlstd{evecmac}\hlkwb{<-}\hlstd{emax}\hlopt{$}\hlstd{evecmax}
\hlstd{ogaxftn}\hlkwb{<-}\hlkwd{geradin}\hlstd{(x, naxftn, ident,} \hlkwc{AA}\hlstd{=}\hlnum{1}\hlstd{,} \hlkwc{BB}\hlstd{=}\hlnum{1}\hlstd{,} \hlkwc{control}\hlstd{=}\hlkwd{list}\hlstd{(}\hlkwc{trace}\hlstd{=}\hlnum{FALSE}\hlstd{))}
\hlstd{gvec}\hlkwb{<-}\hlstd{ogaxftn}\hlopt{$}\hlstd{x}
\hlstd{gval}\hlkwb{<-} \hlopt{-}\hlstd{ogaxftn}\hlopt{$}\hlstd{RQ}
\hlstd{gvec}\hlkwb{<-}\hlkwd{sign}\hlstd{(gvec[[}\hlnum{1}\hlstd{]])}\hlopt{*}\hlstd{gvec}\hlopt{/}\hlkwd{sqrt}\hlstd{(}\hlkwd{as.numeric}\hlstd{(}\hlkwd{crossprod}\hlstd{(gvec)))}
\hlstd{diff}\hlkwb{<-}\hlstd{gvec}\hlopt{-}\hlstd{evecmax}
\hlkwd{cat}\hlstd{(}\hlstr{"Geradin diff eigenval from eigen result: "}\hlstd{,gval}\hlopt{-}\hlstd{evalmax,}\hlstr{"   max(abs(vector diff))="}\hlstd{,}
      \hlkwd{max}\hlstd{(}\hlkwd{abs}\hlstd{(diff)),} \hlstr{"\textbackslash{}n"}\hlstd{)}
\end{alltt}
\begin{verbatim}
## Geradin diff eigenval from eigen result:  -1.435319e-08    max(abs(vector diff))= 4.691479e-07
\end{verbatim}
\end{kframe}
\end{knitrout}

\section{Perspective}

We can compare the different approaches by looking at the ratio of the best
solution time for each method (compiled or interpreted, with best choice of
function) to the time for the Geradin approach for the different matrix sizes. 
In this we will ignore the fact that some approaches do not build the matrix.

\begin{knitrout}\scriptsize
\definecolor{shadecolor}{rgb}{0.969, 0.969, 0.969}\color{fgcolor}\begin{kframe}


{\ttfamily\noindent\bfseries\color{errorcolor}{\#\# Error in plot.window(...): need finite 'ylim' values}}\end{kframe}
\includegraphics[width=\maxwidth]{figure/persp1-1} 

\end{knitrout}


Note that the conjugate gradients method is about as fast as the Geradin specialized 
method, so the latter may be overkill. The eigenvalue solver is finding all solutions, 
so it uses more time as the size of the problems increase. However, finding all solutions
this way is still faster than finding just one with the \code{spg()} function. Nevertheless,
the latter may still be worth keeping in our toolbox as its projection approach is 
sometimes much quicker to set up and test. Human time is more valuable than computer
time in most situations.


To check the value of the Geradin approach, let us use a much larger problem,
with \code{n=2000}. 

\begin{knitrout}\scriptsize
\definecolor{shadecolor}{rgb}{0.969, 0.969, 0.969}\color{fgcolor}\begin{kframe}


{\ttfamily\noindent\bfseries\color{errorcolor}{\#\# Error in optimr(x, fn = nobj, gr = ngrobj, method = "{}Rcgmin"{}, AA = -AA): could not find function "{}optimr"{}}}

{\ttfamily\noindent\itshape\color{messagecolor}{\#\# Timing stopped at: 0 0 0}}\begin{verbatim}
## Times in seconds
\end{verbatim}


{\ttfamily\noindent\bfseries\color{errorcolor}{\#\# Error in cat("{}Build ="{}, t2000b, "{} eigen():"{}, t2000e, "{}\ \ Rcgminu:"{}, t2000c, : object 't2000c' not found}}

{\ttfamily\noindent\bfseries\color{errorcolor}{\#\# Error in cat("{}Ratios: build="{}, t2000b/t2000g, "{} eigen="{}, t2000e/t2000g, : object 't2000c' not found}}\end{kframe}
\end{knitrout}


\section{Conclusions}

The Rayleigh Quotient minimization approach to eigensolutions has an 
intuitive appeal and seemingly offers an interesting optimization 
test problem, especially if we can make it computationally efficient.
To improve time efficiency, we can apply the \R byte code compiler, 
use a Fortran (or other compiled language) subroutine, and choose
how we set up our objective functions and gradients. To improve
memory use, we can consider using a matrix implicitly.

From the tests in this vignette, here is what we may say about these
attempts, which we caution are based on a relatively small sample of
tests:

\begin{itemize}
\item{The \R byte code compiler offers a useful gain in speed when 
our code has statements that access array elements rather than uses
them in vectorized form.}
\item{The \code{crossprod()} function is very efficient.}
\item{Fortran is not very difficult to use for small subroutines
that compute a function such as the implicit matrix-vector product, 
and it allows efficient computations for such operations.}
\item{The \code{eigen()} routine is a highly effective tool for
computing all eigensolutions, even of a large matrix. It does,
however, require the explicit full matrix.}
\item{It is only
worth computing a single solution when the matrix is very large,
in which case a specialized method such as that of Geradin makes
sense and offers significant savings, especially when combined with
the Fortran implicit matrix-product routine. If such a specialized
code is unavailable, a general conjugate gradients code can be quite
competitive for minimizing the Rayleigh Quotient when the other
speed improvements are applied. Both approaches save memory when
an implicit matrix-vector product is used.}
\end{itemize}

\section*{Acknowledgements}

This vignette originated due to a problem suggested by Gabor Grothendieck. Ravi Varadhan
has provided insightful comments and some vectorized functions which
greatly altered some of the observations.

\bibliographystyle{chicago} %The style you want to use for references.
\bibliography{RQ} %The files containing all the articles and 


\section*{Appendix 1: Geradin routine}

\begin{knitrout}\scriptsize
\definecolor{shadecolor}{rgb}{0.969, 0.969, 0.969}\color{fgcolor}\begin{kframe}
\begin{alltt}
\hlstd{ax}\hlkwb{<-}\hlkwa{function}\hlstd{(}\hlkwc{x}\hlstd{,} \hlkwc{AA}\hlstd{)\{}
   \hlstd{u}\hlkwb{<-}\hlkwd{as.numeric}\hlstd{(AA}\hlopt\hlstd{x)}
\hlstd{\}}
\hlstd{bx}\hlkwb{<-}\hlkwa{function}\hlstd{(}\hlkwc{x}\hlstd{,} \hlkwc{BB}\hlstd{)\{}
   \hlstd{v}\hlkwb{<-}\hlkwd{as.numeric}\hlstd{(BB}\hlopt\hlstd{x)}
\hlstd{\}}
\hlstd{geradin}\hlkwb{<-}\hlkwa{function}\hlstd{(}\hlkwc{x}\hlstd{,} \hlkwc{ax}\hlstd{,} \hlkwc{bx}\hlstd{,} \hlkwc{AA}\hlstd{,} \hlkwc{BB}\hlstd{,} \hlkwc{control}\hlstd{=}\hlkwd{list}\hlstd{(}\hlkwc{trace}\hlstd{=}\hlnum{TRUE}\hlstd{,} \hlkwc{maxit}\hlstd{=}\hlnum{1000}\hlstd{))\{}
\hlcom{# Geradin minimize Rayleigh Quotient, Nash CMN Alg 25}
\hlcom{# print(control)}
  \hlstd{trace}\hlkwb{<-}\hlstd{control}\hlopt{$}\hlstd{trace}
  \hlstd{n}\hlkwb{<-}\hlkwd{length}\hlstd{(x)}
  \hlstd{tol}\hlkwb{<-}\hlstd{n}\hlopt{*}\hlstd{n}\hlopt{*}\hlstd{.Machine}\hlopt{$}\hlstd{double.eps}\hlopt{^}\hlnum{2}
  \hlstd{offset}\hlkwb{<-}\hlnum{1e+5} \hlcom{# equality check offset}
  \hlkwa{if} \hlstd{(trace)} \hlkwd{cat}\hlstd{(}\hlstr{"geradin.R, using tol="}\hlstd{,tol,}\hlstr{"\textbackslash{}n"}\hlstd{)}
  \hlstd{ipr}\hlkwb{<-}\hlnum{0} \hlcom{# counter for matrix mults}
  \hlstd{pa}\hlkwb{<-}\hlstd{.Machine}\hlopt{$}\hlstd{double.xmax}
  \hlstd{R}\hlkwb{<-}\hlstd{pa}
  \hlstd{msg}\hlkwb{<-}\hlstr{"no msg"}
\hlcom{# step 1 -- main loop}
  \hlstd{keepgoing}\hlkwb{<-}\hlnum{TRUE}
  \hlkwa{while} \hlstd{(keepgoing) \{}
    \hlstd{avec}\hlkwb{<-}\hlkwd{ax}\hlstd{(x, AA); bvec}\hlkwb{<-}\hlkwd{bx}\hlstd{(x, BB); ipr}\hlkwb{<-}\hlstd{ipr}\hlopt{+}\hlnum{1}
    \hlstd{xax}\hlkwb{<-}\hlkwd{as.numeric}\hlstd{(}\hlkwd{crossprod}\hlstd{(x, avec));}
    \hlstd{xbx}\hlkwb{<-}\hlkwd{as.numeric}\hlstd{(}\hlkwd{crossprod}\hlstd{(x, bvec));}
    \hlkwa{if} \hlstd{(xbx} \hlopt{<=} \hlstd{tol) \{}
       \hlstd{keepgoing}\hlkwb{<-}\hlnum{FALSE} \hlcom{# not really needed}
       \hlstd{msg}\hlkwb{<-}\hlstr{"avoid division by 0 as xbx too small"}
       \hlkwa{break}
    \hlstd{\}}
    \hlstd{p0}\hlkwb{<-}\hlstd{xax}\hlopt{/}\hlstd{xbx}
    \hlkwa{if} \hlstd{(p0}\hlopt{>}\hlstd{pa) \{}
       \hlstd{keepgoing}\hlkwb{<-}\hlnum{FALSE} \hlcom{# not really needed}
       \hlstd{msg}\hlkwb{<-}\hlstr{"Rayleigh Quotient increased in step"}
       \hlkwa{break}
    \hlstd{\}}
    \hlstd{pa}\hlkwb{<-}\hlstd{p0}
    \hlstd{g}\hlkwb{<-}\hlnum{2}\hlopt{*}\hlstd{(avec}\hlopt{-}\hlstd{p0}\hlopt{*}\hlstd{bvec)}\hlopt{/}\hlstd{xbx}
    \hlstd{gg}\hlkwb{<-}\hlkwd{as.numeric}\hlstd{(}\hlkwd{crossprod}\hlstd{(g))} \hlcom{# step 6}
    \hlkwa{if} \hlstd{(trace)} \hlkwd{cat}\hlstd{(}\hlstr{"Before loop: RQ="}\hlstd{,p0,}\hlstr{" after "}\hlstd{,ipr,}\hlstr{" products, gg="}\hlstd{,gg,}\hlstr{"\textbackslash{}n"}\hlstd{)}
    \hlkwa{if} \hlstd{(gg}\hlopt{<}\hlstd{tol) \{} \hlcom{# step 7}
       \hlstd{keepgoing}\hlkwb{<-}\hlnum{FALSE} \hlcom{# not really needed}
       \hlstd{msg}\hlkwb{<-}\hlstr{"Small gradient -- done"}
       \hlkwa{break}
    \hlstd{\}}
    \hlstd{t}\hlkwb{<-} \hlopt{-}\hlstd{g} \hlcom{# step 8}
    \hlkwa{for} \hlstd{(itn} \hlkwa{in} \hlnum{1}\hlopt{:}\hlstd{n) \{} \hlcom{# major loop step 9}
       \hlstd{y}\hlkwb{<-}\hlkwd{ax}\hlstd{(t, AA); z}\hlkwb{<-}\hlkwd{bx}\hlstd{(t, BB); ipr}\hlkwb{<-}\hlstd{ipr}\hlopt{+}\hlnum{1} \hlcom{# step 10}
       \hlstd{tat}\hlkwb{<-}\hlkwd{as.numeric}\hlstd{(}\hlkwd{crossprod}\hlstd{(t, y))} \hlcom{# step 11}
       \hlstd{xat}\hlkwb{<-}\hlkwd{as.numeric}\hlstd{(}\hlkwd{crossprod}\hlstd{(x, y))}
       \hlstd{xbt}\hlkwb{<-}\hlkwd{as.numeric}\hlstd{(}\hlkwd{crossprod}\hlstd{(x, z))}
       \hlstd{tbt}\hlkwb{<-}\hlkwd{as.numeric}\hlstd{(}\hlkwd{crossprod}\hlstd{(t, z))}
       \hlstd{u}\hlkwb{<-}\hlstd{tat}\hlopt{*}\hlstd{xbt}\hlopt{-}\hlstd{xat}\hlopt{*}\hlstd{tbt}
       \hlstd{v}\hlkwb{<-}\hlstd{tat}\hlopt{*}\hlstd{xbx}\hlopt{-}\hlstd{xax}\hlopt{*}\hlstd{tbt}
       \hlstd{w}\hlkwb{<-}\hlstd{xat}\hlopt{*}\hlstd{xbx}\hlopt{-}\hlstd{xax}\hlopt{*}\hlstd{xbt}
       \hlstd{d}\hlkwb{<-}\hlstd{v}\hlopt{*}\hlstd{v}\hlopt{-}\hlnum{4}\hlopt{*}\hlstd{u}\hlopt{*}\hlstd{w}
       \hlkwa{if} \hlstd{(d}\hlopt{<}\hlnum{0}\hlstd{)} \hlkwd{stop}\hlstd{(}\hlstr{"Geradin: imaginary roots not possible"}\hlstd{)} \hlcom{# step 13}
       \hlstd{d}\hlkwb{<-}\hlkwd{sqrt}\hlstd{(d)} \hlcom{# step 14}
       \hlkwa{if} \hlstd{(v}\hlopt{>}\hlnum{0}\hlstd{) k}\hlkwb{<-}\hlopt{-}\hlnum{2}\hlopt{*}\hlstd{w}\hlopt{/}\hlstd{(v}\hlopt{+}\hlstd{d)} \hlkwa{else} \hlstd{k}\hlkwb{<-}\hlnum{0.5}\hlopt{*}\hlstd{(d}\hlopt{-}\hlstd{v)}\hlopt{/}\hlstd{u}
       \hlstd{xlast}\hlkwb{<-}\hlstd{x} \hlcom{# NOT as in CNM -- can be avoided with loop}
       \hlstd{avec}\hlkwb{<-}\hlstd{avec}\hlopt{+}\hlstd{k}\hlopt{*}\hlstd{y; bvec}\hlkwb{<-}\hlstd{bvec}\hlopt{+}\hlstd{k}\hlopt{*}\hlstd{z} \hlcom{# step 15, update}
       \hlstd{x}\hlkwb{<-}\hlstd{x}\hlopt{+}\hlstd{k}\hlopt{*}\hlstd{t}
       \hlstd{xax}\hlkwb{<-}\hlstd{xax}\hlopt{+}\hlkwd{as.numeric}\hlstd{(}\hlkwd{crossprod}\hlstd{(x,avec))}
       \hlstd{xbx}\hlkwb{<-}\hlstd{xbx}\hlopt{+}\hlkwd{as.numeric}\hlstd{(}\hlkwd{crossprod}\hlstd{(x,bvec))}
       \hlkwa{if} \hlstd{(xbx}\hlopt{<}\hlstd{tol)} \hlkwd{stop}\hlstd{(}\hlstr{"Geradin: xbx has become too small"}\hlstd{)}
       \hlstd{chcount}\hlkwb{<-}\hlstd{n} \hlopt{-} \hlkwd{length}\hlstd{(}\hlkwd{which}\hlstd{((xlast}\hlopt{+}\hlstd{offset)}\hlopt{==}\hlstd{(x}\hlopt{+}\hlstd{offset)))}
       \hlkwa{if} \hlstd{(trace)} \hlkwd{cat}\hlstd{(}\hlstr{"Number of changed components = "}\hlstd{,chcount,}\hlstr{"\textbackslash{}n"}\hlstd{)}
       \hlstd{pn}\hlkwb{<-}\hlstd{xax}\hlopt{/}\hlstd{xbx} \hlcom{# step 17 different order}
       \hlkwa{if} \hlstd{(chcount}\hlopt{==}\hlnum{0}\hlstd{) \{}
         \hlstd{keepgoing}\hlkwb{<-}\hlnum{FALSE} \hlcom{# not really needed}
         \hlstd{msg}\hlkwb{<-}\hlstr{"Unchanged parameters -- done"}
         \hlkwa{break}
       \hlstd{\}}
       \hlkwa{if} \hlstd{(pn} \hlopt{>=} \hlstd{p0) \{}
         \hlkwa{if} \hlstd{(trace)} \hlkwd{cat}\hlstd{(}\hlstr{"RQ not reduced, restart\textbackslash{}n"}\hlstd{)}
         \hlkwa{break} \hlcom{# out of itn loop, not while loop (TEST!)}
       \hlstd{\}}
       \hlstd{p0}\hlkwb{<-}\hlstd{pn} \hlcom{# step 19}
       \hlstd{g}\hlkwb{<-}\hlnum{2}\hlopt{*}\hlstd{(avec}\hlopt{-}\hlstd{pn}\hlopt{*}\hlstd{bvec)}\hlopt{/}\hlstd{xbx}
       \hlstd{gg}\hlkwb{<-}\hlkwd{as.numeric}\hlstd{(}\hlkwd{crossprod}\hlstd{(g))}
       \hlkwa{if} \hlstd{(trace)} \hlkwd{cat}\hlstd{(}\hlstr{"Itn"}\hlstd{, itn,}\hlstr{" RQ="}\hlstd{,p0,}\hlstr{" after "}\hlstd{,ipr,}\hlstr{" products, gg="}\hlstd{,gg,}\hlstr{"\textbackslash{}n"}\hlstd{)}
       \hlkwa{if} \hlstd{(gg}\hlopt{<}\hlstd{tol)\{} \hlcom{# step 20}
         \hlkwa{if} \hlstd{(trace)} \hlkwd{cat}\hlstd{(}\hlstr{"Small gradient in iteration, restart\textbackslash{}n"}\hlstd{)}
         \hlkwa{break} \hlcom{# out of itn loop, not while loop (TEST!)}
       \hlstd{\}}
       \hlstd{xbt}\hlkwb{<-}\hlkwd{as.numeric}\hlstd{(}\hlkwd{crossprod}\hlstd{(x,z))} \hlcom{# step 21}
       \hlstd{w}\hlkwb{<-}\hlstd{y}\hlopt{-}\hlstd{pn}\hlopt{*}\hlstd{z} \hlcom{# step 22}
       \hlstd{tabt}\hlkwb{<-}\hlkwd{as.numeric}\hlstd{(}\hlkwd{crossprod}\hlstd{(t,w))}
       \hlstd{beta}\hlkwb{<-}\hlkwd{as.numeric}\hlstd{(}\hlkwd{crossprod}\hlstd{(g,(w}\hlopt{-}\hlstd{xbt}\hlopt{*}\hlstd{g)))}
       \hlstd{beta}\hlkwb{<-}\hlstd{beta}\hlopt{/}\hlstd{tabt} \hlcom{# step 23}
       \hlstd{t}\hlkwb{<-}\hlstd{beta}\hlopt{*}\hlstd{t}\hlopt{-}\hlstd{g}
    \hlstd{\}} \hlcom{# end loop on itn -- step 24}
  \hlstd{\}} \hlcom{# end main loop -- step 25}
  \hlstd{ans}\hlkwb{<-}\hlkwd{list}\hlstd{(}\hlkwc{x}\hlstd{=x,} \hlkwc{RQ}\hlstd{=p0,} \hlkwc{ipr}\hlstd{=ipr,} \hlkwc{msg}\hlstd{=msg)} \hlcom{# step 26}
\hlstd{\}}
\end{alltt}
\end{kframe}
\end{knitrout}
\end{document}} 
